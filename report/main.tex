% This is the main file to setup the document.
% Document organization and appearance settings are all done here
% Each chapter is a separate tex file, all linked together here


%---------------------------- Preamble ------------------------------

% Document type and font --
\documentclass[12pt,a4paper,oneside,french]{book}
\usepackage[utf8]{inputenc} %utf-8 encoding for ASCII symbols
\usepackage[T1]{fontenc}
\usepackage{xcolor}
\usepackage{afterpage}
\usepackage{rotating}
\usepackage[export]{adjustbox}% http://ctan.org/pkg/adjustbox
\usepackage{pdfpages}


% Algorithms
\usepackage{algorithm}
\usepackage{algpseudocode}

%Tables
\usepackage{tabularx}
\usepackage{tablefootnote}

% Document Diagrams
\usepackage{pgf-umlcd}

% figures should not bypass sections
\usepackage[section]{placeins}

% inserting a blank page __
\newcommand\blankpage{%
    \null
    \thispagestyle{empty}%
    \addtocounter{page}{-1}%
    \newpage}
    
\usepackage{parskip}
    
% insert packages here --

\usepackage{graphicx} %for handling images
\usepackage[english]{babel} 
%\usepackage{bbold}
\usepackage{dsfont}
\usepackage{amsmath}  %for math symbols$
\usepackage{amssymb}
\usepackage{mathabx} % font for math symbols
\usepackage{fancyhdr} %constructing headers and footers
\usepackage{breakcites} %to avoid citations extending into the margin
\usepackage{csquotes} %in-line and display quotations
\usepackage[margin=1in]{geometry}  %to reduce margins to 1 inch 
\usepackage{sidecap} %to enable side captions on figures
\usepackage{subcaption}
\usepackage{float}
\usepackage{footnote}
\usepackage[bottom]{footmisc}


\usepackage[notransparent]{svg}
\usepackage{mathrsfs}
\usepackage{pgf-umlcd}
\usepackage{dirtytalk}
\usepackage{tikz} 
\usepackage{lscape}
\usepackage{colortbl}


%Tick & Cross
\usepackage{pifont}
\newcommand{\cmark}{\ding{51}}%
\newcommand{\xmark}{\ding{55}}%

\newcommand{\overbar}[1]{\mkern 1.5mu\overline{\mkern-1.5mu#1\mkern-1.5mu}\mkern 1.5mu}



\DeclareMathOperator{\Adj}{Adj}
\DeclareMathOperator*{\argmin}{\arg\!\min}
\DeclareMathOperator*{\argmax}{\arg\!\max}
\DeclareMathOperator{\sign}{sign}
\DeclareMathOperator{\hardtanh}{hardtanh}
\DeclareMathOperator{\clipfn}{clip}
\DeclareMathOperator{\mat}{mat}
\DeclareMathOperator{\rank}{rank}
\DeclareMathOperator{\End}{End}
\DeclareMathOperator{\Max}{Max}
\DeclareMathOperator{\Min}{Min}
\DeclareMathOperator{\Opt}{Opt}
\DeclareMathOperator{\P0} {\Min}
\DeclareMathOperator{\choice}{choice}
\DeclareMathOperator{\Id}{Id}
\DeclareMathOperator{\ImageSet}{Im}
\DeclareMathOperator{\zip}{zip}
\DeclareMathOperator{\domain}{domain}
\DeclareMathOperator{\pop}{pop}
\DeclareMathOperator{\False}{\textbf{False}}
\DeclareMathOperator{\True}{\textbf{True}}
\DeclareMathOperator{\Not}{\textbf{not}}
\DeclareMathOperator*{\transform}{transform}
\DeclareMathOperator*{\projection}{projection}
\DeclareMathOperator{\arcconsistency}{arc consistency}
\DeclareMathOperator{\CSP}{\mathtt{CSP}}
\DeclareMathOperator{\WeakOptimal}{\mathtt{WeakOptimal}}
\DeclareMathOperator{\StrongOptimal}{\mathtt{StrongOptimal}}
\DeclareMathOperator{\PayoffOptimal}{\mathtt{PayoffOptimal}}
\DeclareMathOperator{\ones}{\mathbf{1}}
\DeclareMathOperator{\UCT}{UCT}
\DeclareMathOperator{\PUCT}{PUCT}
\newtheorem{definition}{Définition}
\newtheorem{remark}{Remarque}
\newtheorem{proof}{Preuve}
\newtheorem{lemma}{Lemme}
%Used for encapsulation
\newcommand{\VertexSet}{\ensuremath{V}}
\newcommand{\EdgeSet}{\ensuremath{E}}
\newcommand{\PlayerSet}{\ensuremath{P}}
\newcommand{\IndefinitePlayer}{\ensuremath{\Opt}}
\newcommand{\Player}{\IndefinitePlayer}
\newcommand{\Dnp}[2]{\mathcal{D}(#1,#2)}
\newcommand{\Dnm}[2]{\mathcal{D}(#1,#2)}
\newcommand{\Dsnp}[2]{\mathcal{D}^S(#1,#2)}
\newcommand{\Dsnm}[2]{\mathcal{D}^S(#1,#2)}
\newcommand{\PowerSet}[1]{\mathscr{P}(#1)}
\newcommand{\Probability}[1]{\mathscr{P}\left(#1\right)}
\newcommand{\ConditionalProbability}[2]{\mathscr{P}\left(#1 \mid #2\right)}
\newcommand{\Expected}[1]{\mathbb{E}\left[#1\right]}
\newcommand{\ConditionalExpected}[2]{\mathbb{E}\left[#1 \mid #2\right]}
\newcommand{\ComRing}{\mathcal{R}}
\newcommand{\Enclose}[1]{\left(#1\right)}
\newcommand{\Distribution}[1]{\mathscr{D}(#1)}


\newcommand{\ProjectTitle}{Implementation, generation, analysis and predictive modeling of mean payoff games using self-play}
\newcommand{\AuthorName}{Rami ZOUARI}


%Middle Align
\usepackage{array,multirow,makecell}
\newcolumntype{C}[1]{>{\arraybackslash}p{#1}}


% header and footer
\usepackage{enumitem}
\setlist{leftmargin=*,itemsep=0pt}

\usepackage{centernot}
%\usepackage[linesnumbered,ruled,vlined,french,onelanguage]{algorithm2e}

\usepackage{quotchap}
\makeatletter
\renewcommand{\@makechapterhead}[1]{
	\chapterheadstartvskip
	{\size@chapter{\sectfont\raggedright
			{\chapnumfont
				\ifnum \c@secnumdepth >\m@ne
				\if@mainmatter\thechapter
				\fi\fi
				\par\nobreak}
			{\raggedright\advance\leftmargin10em\interlinepenalty\@M #1\par}}
		\nobreak\chapterheadendvskip}}
\makeatother
\renewcommand*{\chapterheadendvskip}{\vspace{2cm}}

\geometry{hmargin=2.5cm,vmargin=2.5cm}

\usepackage{fancyhdr}
\pagestyle{fancyplain}
\lhead{\fancyplain{}{\nouppercase{\textit{\leftmark}}}}
\chead{\fancyplain{}{}}
\rhead{\fancyplain{}{}}
\lfoot{\fancyplain{}{}}
\cfoot{\fancyplain{}{}}
\rfoot{\fancyplain{\thepage}{\thepage}}
\renewcommand{\headrulewidth}{1pt}
\renewcommand{\footrulewidth}{1pt}

\renewcommand{\thesection}{\arabic{section}}

\usepackage{titlesec}
\titleformat{\paragraph}{\fontsize{11}{10}\bfseries}{\theparagraph}{1em}{}
\titlespacing*{\paragraph}{0pt}{10pt plus 2pt minus 0pt}{0pt plus 2pt minus 0pt}

\setcounter{secnumdepth}{4}
\setcounter{tocdepth}{4}


\setlength{\parskip}{8pt}
\usepackage{setspace}

\usepackage{url}

% linked table of contents
\usepackage{hyperref}  
% Comment before printing to remove links' colors
\definecolor{darkblue}{rgb}{0.0, 0.0, 0.5}
\hypersetup{
	colorlinks,
	linktocpage=true,
	linkcolor={darkblue},
	citecolor={darkblue},
	urlcolor={blue}}
% Glossaries and acronyms list
\usepackage[acronym,toc,nomain]{glossaries}
\makeglossaries
\newacronym{ny}{NY}{New York}
\newacronym{la}{LA}{Los Angeles}
\newacronym{un}{UN}{United Nations}
%\loadglsentries{glossary}

% Bibliography --

\bibliographystyle{acm}
\usepackage[backend=bibtex]{biblatex}      %use the biblatex package
\usepackage[nottoc,numbib]{tocbibind}
%\addbibresource{biblio.bib}   %path to the bib file
\bibliography{biblio}

% Set path to images
\graphicspath{ {images/} }  
\singlespacing  %making text double spaces


% End of preamble
%----------------------------- Document -----------------------------

%\renewcaptionname{english}{\listfigurename}{Liste des Figures}
%\renewcaptionname{english}{\listtablename}{Liste des Tableaux}
%\renewcaptionname{english}{\contentsname}{Contenu}

\renewcommand\multicitedelim{\addsemicolon\space}

\newtheorem{theorem}{Theorem}

%Chapter numbering
%\renewcommand{\thechapter}{\Roman{chapter}}


\begin{document}
%\pagenumbering{gobble}
% Making title page
\begin{titlepage}
   \begin{center}
   \begin{doublespacing}

       \begin{figure}
       \begin{center}

        \begin{minipage}[c]{0.8\textwidth}
        	\subfloat{{\includegraphics[width=0.3\textwidth]{Figures/dBSense-logo.png}\hspace*{1.25cm}}}
            \subfloat{{\includegraphics[width=0.2\textwidth]{images/insat.png}\hspace*{2cm}}}
            \subfloat{{\includegraphics[width=0.2\textwidth]{Figures/UniversityCarthage-logo.png}}}
        \end{minipage}
        \hfill
        \hfill
        \end{center}
        \end{figure}
       
       {\Large\textbf{Institut National des Sciences Appliquées et des Technologies}\\}
       {\Large\textbf{UNIVERSITE DE CARTHAGE}\\}
       \noindent\rule{15cm}{0.4pt}
       {\Huge\textbf{STAGE INGÉNIEUR}\\}
       {\large\textbf{Génie Logiciel}\\}
       {\Large\textbf{\ProjectTitle}\\}

       %\textbf{DOCTOR OF PHILOSOPHY}
       \vspace{10 mm}
       \vspace{2.5 mm}
       {\huge\textbf{}}
       \noindent\rule{15cm}{0.5pt}
       \vspace{10 mm}
       

        {\Large\textbf{Auteur:}\\}
       {\Large\textbf{\AuthorName}\\}
       \vspace{2.5 mm}
       

     
     \vspace{8 mm}
        
     {\large\textbf{2021/2022}}
    
    \end{doublespacing}

   \end{center}
\end{titlepage}

\newpage
\chapter*{Dedication}

To my beloved parents, Your unwavering love, countless sacrifices, and boundless support have shaped me into the person I am today. I am forever grateful for the countless sacrifices you made to provide me with opportunities and a better life. Your guidance and encouragement have
been my guiding lights, and I cherish every moment spent with you. Though I can never
fully repay you, know that your love and sacrifices will always be etched in my heart.
\\
\\
To my family,
Your love, encouragement, and solidarity have been my bedrock. Your support,
both seen and unseen, has been a driving force behind my accomplishments. Each one of
you has played an invaluable role in my life, and I am forever grateful for your presence
and support.
\\
\\
To my dear friends,
The friendships forged on this journey have been priceless treasures. Your laughter, companionship, and unwavering support have enriched my life in ways words cannot
express. The memories we’ve created together are etched in my heart, and I am honored
to dedicate this book to the unforgettable times we’ve shared.
With heartfelt thanks,
Oussama
\chapter*{Aknowledgment}

Words cannot express my gratitude to Prof. Riadh Robbana, my academic supervisor, for his excellent mentorship not only throughout my thesis journey, but also my academic journey.

I am sincerely thankful to Prof. Manuel Bodirsky, my supervisor at TU Dresden, for
providing me with the invaluable opportunity to research . His expertise, insight and kindness boosted my learning
experience and expanded my horizons.

I am greatly thankful to Mr. Florian Starke, my co-supervisor at TU Dresden, for not only providing guidance, but also for being a great friend.

I would also like to express my recognition to my esteemed professors whose excellent courses and dedicated mentorship have equipped me with the knowledge and
skills necessary for this achievement. Their commitment to education has been instrumental in my academic growth.

I am indebted to the INSAT community for fostering an environment of
inspiration and growth. The collective spirit of students, faculty, and staff has continuously motivated me to strive for excellence.

Through this acknowledgment, I express my heartfelt gratitude to my friends who have been my pillars of strength.

Lastly, I would be remiss in not mentioning my family, my father Anouar, my mother Narjes and my two sisters Souha and Safa, Their belief in me has kept my spirits and motivation high during this process. I would also like to thank my cat Lara for all the entertainment and emotional support during the writing of this thesis.
\\
\\
\\
With heartfelt thanks,

Rami.





\chapter*{Abstract}
\acrfullpl{mpg} are being extensively investigated with the hopes of finding a general polynomial time solver. While this is a very desired goal, it is ambitious. We will use instead a \acrfull{rl} method using an underlying \acrfull{dl} model based on \acrfullpl{gnn} that tries to approximate solutions of a \acrshort{mpg} instance. The learnable parameters of the model will be updated using a \acrfull{sp} approach that is inspired from Alpha Zero.

To achieve all this, we will implement a \acrshort{mpg} library, generate two \acrshort{mpg} datasets, and then try solve and analyse them.  We will then design our \acrshort{gnn} model with high focus on keeping as much symmetries as possible in the model itself. Finally, we will implement a whole distributed \acrshort{sp} system based on Alpha Zero that will be used for learning purposes, in the hope of getting a model that plays decently. 

% Roman page numbering to start from abstract onwards

% Main matter starts here --
% Inserting individual chapters. Mention chapter titles here and simple link the chapter's tex file

\hypersetup{linkcolor=black}

\tableofcontents


\listoffigures
\listoftables
\listofalgorithms
\addcontentsline{toc}{chapter}{List of Algorithms}
\printglossary[type=\acronymtype]
\chapter*{Introduction}

\addcontentsline{toc}{chapter}{Introduction}

With the rising power of \acrfull{ai}, and especially \acrfull{dl}, \acrshort{ai} is being applied with a huge success to problems that were computationnally intractable.

In this report, we will focus on a family of games known as \acrfullpl{mpg}. It has been studied and analysed extensively with the hopes of finding an efficient general method to solve instances of that game. To the day of writing this report, no breakthroughs were being found. 

Instead, we will try to approximate solutions using a completely different method based on \acrfull{rl}, with an underlying \acrshort{dl} algorithm using \acrfull{sp} to refine its understanding of the game. Our methods will not only learn individual instances of the game, but also generalise to a whole class of \acrshortpl{mpg}

In the first chapter, we will give a presentation about the host insitute, an informal introduction to \acrshortpl{mpg} and talk about the state of the art in \acrshort{mpg} theory, its relation to other fields, and the existing \acrfull{ml} methods in related games.

In the second chapter, we will formalise \acrshort{mpg} using known results from the existing theory. Then, we will design and implement a \acrshort{mpg} library that is based on the theoretical frameworks. This library will be used by default in all the remaining of the project.

In the third chapter, we will generate two datasets of \acrshort{mpg} using efficient implementations of graph generation. We will then try to solve all instances of both datasets using our most efficient implementation of \acrshort{mpg} solver based on \acrfull{csp} methods.  We will use also a \acrfull{hpc} cluster to further minimise the execution time.

In the fourth chapter, we will analyse briefly both datasets with their annotations, and try to confirm empirically our hypotheses about \acrshort{mpg}. We will then design a \acrshort{gnn} model that conforms to some desired properties, based in our findings.

In the fifth and final chapter, we will implement a whole distributed \acrshort{rl} system that uses \acrfull{sp} that refines the learnable parameters of the model discussed in chapter \ref{chapter:ModelDesign}.  We will base the full architecture of the system in the famous Alpha Zero paper \cite{AlphaZero}. Finally, we will deploy the system in a large \acrshort{hpc} cluster and execute it for refining the model.


\chapter{Cadre du Stage }

\section*{Introduction}


\chapter{Formalisation \& Implementation}
\section{Introduction}

\section{Formalisation}
To define a Mean Payoff Game, we will start by formalising a weighted di-graph\footnote{Directed Graph}.

\subsection{Di-Graph}
A Weighted Di-Graph $G$ is a tuple $(\VertexSet,\EdgeSet,W)$ where:

\begin{itemize}
		\item $\VertexSet$ is the set of vertices.
		\item $\EdgeSet \subseteq \VertexSet\times \VertexSet$ is the set of edges.
		\item $W:\EdgeSet\rightarrow \mathbb{G}$ is the weight function, assigning a weight for every edge, with $\mathbb{G}$ some ordered abelian group\footnote{This definition is too general. We will only consider $\mathbb{G}\in \{\mathbb{Z},\mathbb{Q},\mathbb{R}\}.$ Also, $\mathbb{G}$ itself should be clear from the context.}. 
\end{itemize}
\subsection{Mean Payoff Game}
Formally, a \textbf{Mean Payoff Graph} is a tuple $(\VertexSet,\EdgeSet,W,\PlayerSet,s,p)$ where:
\begin{itemize}
	\item $\mathcal{G}=(\VertexSet,\EdgeSet,W)$ is a di-graph.
		\item $s\in \VertexSet$ denotes the starting position.
	\item $\PlayerSet=\{\text{Max},\text{Min}\}$ is the set of players.
	\item $p\in \PlayerSet$ the starting player
\end{itemize}


A  \textbf{Mean Payoff Game} is a perfect information, zero-sum, turn based game played indefinitively on a Mean Payoff Graph as follow:
\begin{itemize}
\item The game starts at $u_0=s$, with player $p_0=p$ starting.
\item For each $n\in\mathbb{N},$ Player $p_n$ will choose a vertex $u_{n+1}\in \Adj u_n,$ with a payoff $w_n=W(u_n,u_{n+1})$
\item The winner of the game will be determined by the Mean Payoff. There are different winning conditions.
\end{itemize}

\begin{table}[h]
	\small
	\begin{tabularx}{\textwidth}{| p{2cm} | X | X | X |}
		\hline
		
		Name & $\Max$ winning criteria & $\Min$ winning criteria & Draw criteria  \\
		\hline
		$C_1$ & \begin{equation*}
			\liminf_{n\in\mathbb{N}^*} \frac{1}{n}\sum_{k=0}^{n-1} w_k \ge 0
		\end{equation*} & \begin{equation*}
		\liminf_{n\in\mathbb{N}^*} \frac{1}{n}\sum_{k=0}^{n-1} w_k < 0
		\end{equation*} & \cellcolor{gray!75} \\
		\hline
		$C_2$ & \begin{equation*}
			\liminf_{n\in\mathbb{N}^*} \frac{1}{n}\sum_{k=0}^{n-1} w_k > 0
		\end{equation*} & \begin{equation*}
			\liminf_{n\in\mathbb{N}^*} \frac{1}{n}\sum_{k=0}^{n-1} w_k < 0
		\end{equation*} & \begin{equation*}
		\liminf_{n\in\mathbb{N}^*} \frac{1}{n}\sum_{k=0}^{n-1} w_k = 0
		\end{equation*} \\
		\hline
		 $C_3$ & \begin{equation*}
		 	\liminf_{n\in\mathbb{N}^*} \frac{1}{n}\sum_{k=0}^{n-1} w_k > 0
		 \end{equation*} & \begin{equation*}
		 	\limsup_{n\in\mathbb{N}^*} \frac{1}{n}\sum_{k=0}^{n-1} w_k < 0
		 \end{equation*} & \begin{equation*}
		 \begin{cases} 
		 	\displaystyle \liminf_{n\in\mathbb{N}^*} \frac{1}{n}\sum_{k=0}^{n-1} w_k \le 0 \\
		 	 \displaystyle \limsup_{n\in\mathbb{N}^*} \frac{1}{n}\sum_{k=0}^{n-1} w_k \ge 0
		 \end{cases}
		 \end{equation*}\\
		\hline
	\end{tabularx}
	\caption{Winning conditions for Mean Payoff Games
		\label{table:WinningConditions}}
\end{table}
Here, table \ref{table:WinningConditions} gives the different winning criteria that we have considered:
\begin{itemize}
	\item $C_1$ was used in \cite{MPGMaxAtom} to calculate the optimal strategy for player $\Max$
	\item $C_2$ is modification of $C_1$ that introduces the possibility of drawing.
	\item $C_3$ is symmetric\footnote{It does not give an advantage towards any player.}, and will be used for our machine learning. It was referenced in \cite{TropicalCSP}.
\end{itemize}
Now, there is a slight difference between the three winning conditions.
\newline For example, an optimal strategy with respect to $C_1$ may not be optimal with respect to $C_2$. While this can happen, it is unlikely.
\newline What about $C_2$ and $C_3$? As different as they appear, they are equivalent in the scope of this report\footnote{They are still different conditions in general.}.
This is demonstrated in \ref{section:Formalisation:MeanPayoff}.
\subsection{Well Foundness}
It is not very clear from the definition that the game is well founded. \newline
In fact, there are choices for which the mean payoff does not converge. That is the sequence $\left(\frac{1}{n}\sum_{k=0}^{n-1} w_k \right)_{n\in\mathbb{N}^*}$ does not converge. \newline One such example is the sequence defined by:
$$
w_n=(-1)^{\lfloor  \log_2 (n+1)\rfloor}
$$
For that sequence, the $(2^r-1)$-step mean payoff is equal to:
\begin{align*}
	\sum_{k=0}^{2^r-2} w_k &= 	\sum_{k=1}^{2^r-1}(-1)^{\lfloor  \log_2 (k)\rfloor} = \sum_{i=0}^{r-1}\sum_{j=2^{i}}^{2^{i+1}-1}(-1)^{\lfloor  \log_2 (j)\rfloor} \\
	&=\sum_{i=0}^{r-1}\sum_{j=2^{i}}^{2^{i+1}-1}(-1)^i =\sum_{i=0}^{r-1}(2^{i+1}-2^i)(-1^i) \\
	&=\sum_{i=0}^{r-1}(-2)^i = \frac{1-(-2)^r}{3} \\
	\implies \frac{1}{2^r-1}\sum_{k=0}^{2^r-2} w_k  &= \frac{1}{3} \cdot \frac{1-(-2)^r}{2^r-1} = \frac{1}{3} \cdot \frac{2^{-r}-(-1)^r}{1-2^{-r}}
\end{align*}
That sequence has two accumulation points $\pm \frac{1}{3},$ and thus, it does not converge.

On the other hand, the introduction of the supremum and infimum operators in the table \ref{table:WinningConditions} will solve the convergence problem, as the resulting sequences will become monotone.

An example of an execution that gives a rise to such payoffs is the following Meab Payoff Game instance\footnote{Note that the proposed pair of strategies is odd in the sense that it appears that both players cooperated on the construction of non-convergent mean payoffs instead of trying ot win the game.}:
\begin{figure}[H]
	\centering
	\begin{subfigure}[b]{0.45\textwidth}
		\raggedleft
		\begin{tikzpicture}[->,>=stealth',shorten >=1pt,auto,node distance=4cm,
			thick,main node/.style={circle,draw,font=\Large\bfseries}]
			\node[main node, fill=gray!50] (1) {$0$}; 
			\node[main node] (2) [right of =1] {$1$}; 
			\path (1) edge [loop above] node {1} (1)
			edge [bend right] node [below] {-1} (2)
			(2) edge [bend right] node [above] {1} (1)
			edge [loop above] node {-1} (2);
		\end{tikzpicture} 
		\caption{Representation of the Mean Payoff Game}
	\end{subfigure}
  \hfill
	\begin{subfigure}[b]{0.45\textwidth}
		\raggedright
		\small
		Pair of strategies defined as:
		\begin{align*}
		\Phi: &V^+ \times P \rightarrow V \\
		 &(s_0\dots s_r, p) \rightarrow B(r)\bmod 2
		\end{align*}
		\scriptsize
		With $B(r)$ the position of the left-most bit in the binary representation of $r$
		\caption{Definition of both strategies}
	\end{subfigure}
	\caption{An example of an execution with non-convergent Mean Payoffs
			\label{fig:MeanPayoffNonConvergence}}
\end{figure}
\subsection{Properties}
A Mean Payoff Game has many properties that are interesting from a game theory perspective.
\subsubsection{Two Player}
The game is a two player game.
\subsubsection{Turn Based}

\subsection{Symmetries}
Mean Payoff Games exhibits many natural symmetries.
\subsubsection{Duality}
The main symmetry is the duality between 
$\Max$ and $\Min.$
\newline For this we will define the dual $\bar{G}$ of a mean payoff game $G=(\VertexSet,\EdgeSet,W\PlayerSet,s,p)$ as the following:
$$
\bar{G}= (\VertexSet,\EdgeSet,-W,s,\bar{p})
$$
This duality is important due to the following theorem.
\begin{theorem}
	For every mean payoff game $G$, the objective of player $\Max$ is equivalent to the objective of player $\Min$ in $\bar{G}.$
\end{theorem}
With that, there is not any major difference between $\Max$ and $\Min$ from a theoretical point of view.
\newline In fact, without any loss of generality, we can assume that $\Max$ is the starting player. And this is what we will do by default in this report.
\begin{figure}
	\begin{tikzpicture}[->,>=stealth',shorten >=1pt,auto,node distance=2cm,
		thick,main node/.style={circle,draw,font=\Large\bfseries}]
		\node[main node, fill=gray!50] (1) {$0$}; 
		\node[main node] (2) [above of =1] {$1$}; 
		\node[main node] (3) [above right of =2] {$2$}; 
		\node[main node] (4) [above of =3] {$3$}; 
		\node[main node] (5) [above left of =2] {$4$}; 
		\node[main node] (6) [above of =5] {$5$}; 
		\node[main node] (7) [above of =6]{$6$}; 
		\path (1) edge node {2} (2)
		(2) edge node {3} (3)
			edge node {3} (5)
		(3) edge [bend left] node {-5} (4)
			edge [bend left] node {-5} (2)
		(4) edge [bend left] node {-5} (2)
			edge [bend left] node {-2} (4)
		(5) edge node {3} (6)
			edge [bend left=25] node {3} (7)
		(6) edge node {3} (1)
		(7) edge [bend right=45] node {3} (1);
	\end{tikzpicture} 
\end{figure}

\subsection{Strategy}
\subsubsection{Deterministic Strategies}
Let $p$ be a player. \newline 
A (deterministic) strategy is a function $\Pi^{p}:\VertexSet^+\rightarrow \VertexSet$ such that:
$$
\forall v_0\dots v_r\in \VertexSet^+, \quad \Pi_p(v_0\dots v_r) \in \Adj v
$$	
If the strategy does only depend on the current vertex, we say it is a memoryless (deterministic) strategy.  $\Pi:\VertexSet\rightarrow \VertexSet$
\newline In this report, we will use the term positional strategies as an alias for memoryless deterministic strategies, which is conforming to the established litterature of mean payoff games.
\newline Positional strategies are crucial for our analysis as a result of the following theorem.
\begin{theorem}
	\label{theorem:OptimalStrategy}
	For all Mean Payoff Games, each player has an optimal positional strategy.
\end{theorem}

\subsubsection{Probabilistic Strategies}
A probabilistic strategy is a random process that assigns for each sequence of vertices $v\in\mathcal{V}$ a probability distribution over $\Adj v.$ This constitutes the most general strategy of a player:
$$
\forall v_0\dots v_r\in \VertexSet^+, \quad \Pi_p(v_0\dots v_r) \in \Distribution{\Adj v}
$$
\subsubsection{Considered Strategies}
Strategies that depends in complete past histories are in general intractable. For Mean Payoff Game, it is proven that the optimal strategy is a \textbf{deterministic} and \textbf{memoryless}.
\newline For that we will only consider \textbf{memoryless} strategies. And for the scope of this report:
\begin{itemize}
	\item A deterministic strategy should refer to memoryless deterministic strategy.
	\item A probabilistic strategy should refer to memoryless probabilistic strategy.
	\item A strategy should refer to memoryless deterministic strategy.
\end{itemize}
We will still consider (memoryless) probabilistic strategies as they reside in a smooth space, and thus they can be used for machine learning purposes.
\subsubsection{Deterministic Optimal Strategy}
There are three kinds of optimality:
\paragraph{Weak Optimality}: In the deterministic case, a strategy $\Phi$ of player $p\in \PlayerSet$ is weakly optimal if one of the following is true:
\begin{itemize}
	\item For each strategy $\Phi^{p}$ of player $p,$ player $\bar{p}$ can win the game by finding a countering strategy $\Phi^{\bar{p}}.$
	\item Player $p$ will not lose the game no matter his opponent's strategy
\end{itemize}

\paragraph{Strong Optimality}: In the deterministic case, a strategy $\Phi$ of player $p\in \PlayerSet$ is strongly optimal if one of the following is true:
\begin{itemize}
	\item For each strategy $\Phi^{p}$ of player $p,$ player $\bar{p}$ can win or tie the game by finding a countering strategy $\Phi^{\bar{p}}.$
	\item Player $p$ will win the game no matter his opponent's strategy
\end{itemize}

\paragraph{Payoff Optimality}: In the deterministic case, a strategy $\Phi$ of player $p\in \PlayerSet$ is payoff optimal if independently of $\bar{p}$'s strategy it:
\begin{itemize}
	\item Maximises the Mean Payoff if $p=\text{Min}$ 
	\item Minimises the Mean Payoff otherwise
\end{itemize}
Now we have the following hiearchy considering the set of optimal strategies:
$$
\forall \text{Mean Payoff Game}\ G,\forall p\in \PlayerSet,\quad \PayoffOptimal(G,p) \subseteq \StrongOptimal(G,p) \subseteq \WeakOptimal(G,p)
$$
\subsection{Mean Payoff}
\label{section:Formalisation:MeanPayoff}
We have used the word ``Mean Payoff" extensively, and they are the central entity in mean payoff games\footnote{This explains the name ``mean payoff game".} but we still did not define it.
\newline We had to delay the definition as it requires the knowledge of the mechanics of the game, and how strategies work. This section will define and formalize the mean payoff, and highlights its relevance.
\subsubsection{Mean Payoff}
For a mean payoff game $G$, with a deterministic pair of strategies $(\Phi^{\Max},\Phi^{\Min}),$ we will define two terms $v^+(G,\Phi^{\Max},\Phi^{\Min})$ and $v^-(G,\Phi^{\Max},\Phi^{\Min})$ as follow\footnote{For probabilistic strategies, both terms are random variables, so we will be interested in their expected values.}:
\begin{align*}
v^+(G,\Phi^{\Max},\Phi^{\Min}) &=\limsup_{n\in\mathbb{N}^*} \frac{1}{n}\sum_{k=0}^{n-1} w_k \\
v^-(G,\Phi^{\Max},\Phi^{\Min}) &=\liminf_{n\in\mathbb{N}^*} \frac{1}{n}\sum_{k=0}^{n-1} w_k
\end{align*} 
Where $(w_n)_{n\in\mathbb{N}}$ is the sequence of payoffs generated by the instance.
\newline These two terms were used in table $\ref{table:WinningConditions}$ when discussing the winning conditions, and we will call them respectively the supremum mean payoff, and the infimum mean payoff. 
\begin{theorem}
	For every mean payoff game $G$, and every pair of strategies $(\Phi^{\Max},\Phi^{\Min})$, both the supremum mean payoff $v^+(G,\Phi^{\Max},\Phi^{\Min})$ and the infimum mean payoff $v^-(G,\Phi^{\Max},\Phi^{\Min})$ are guaranteed to exist, and:
	\begin{equation}
		\label{eqn:InfSupRelationMeanPayoff}
		v^-(G,\Phi^{\Max},\Phi^{\Min}) \le v^+(G,\Phi^{\Max},\Phi^{\Min})
	\end{equation}
\end{theorem}
If both terms are equal in equation  \eqref{eqn:InfSupRelationMeanPayoff}, we say that the game instance has a mean payoff $v(G,\Phi^{\Max},\Phi^{\Min})=v^-(G,\Phi^{\Max},\Phi^{\Min})=v^+(G,\Phi^{\Max},\Phi^{\Min}).$

\subsubsection{Convergence Issues}
This mean payoff itself is the heart of mean payoff games. A major problem lies in the fact that the existence of a mean payoff is not guaranteed, and an example for such case is found in the \newline figure \ref{fig:MeanPayoffNonConvergence}.
\newline Now, fortunately, this is not an issue, as we are interested in good strategies. Theorem \ref{theorem:OptimalStrategy} proven in \cite{PositionalStrategies} states that for each player, the set of optimal strategies contains a positional one. This is a very important result as we can limit the domain of our optimization problem\footnote{The problem of finding optimal strategies} to the more tractable positional\footnote{memoryless and deterministic} and behavioural\footnote{memoryless and probabilistic} strategies.
\newline We start by tackling the existence of the mean payoff for positional strategies in the the following theorem:
\begin{theorem}
	\label{theorem:MeanPayoffExistence}
	For every mean payoff $G$, and a pair of positional strategies $(\Phi^{\Max},\Phi^{\Min}),$ the mean payoff $v(G,\Phi^{\Max},\Phi^{\Min})$ exists.
\end{theorem}
We will give a constructive proof of theorem \ref{theorem:MeanPayoffExistence} in section \ref{section:StrategyEvalution}. We will also provide a linear algorithm for its calculation.
\newline Furthermore, in our machine learning model, we will approach the problem using behavioural strategies. Also luckily, its existence is guaranteed by the following theorem:
\begin{theorem}
	\label{theorem:MeanPayoffExistenceProbabilstic}
	For every mean payoff $G$, and a pair of behavioural strategies $(\Pi^{\Max},\Pi^{\Min}),$ the expected mean payoff $\mathbb{E}[v(G,\Pi^{\Max},\Pi^{\Min})]$ exists.
\end{theorem}
Unlike theorem \ref{theorem:MeanPayoffExistence} theorem \ref{theorem:MeanPayoffExistenceProbabilstic} was very challenging to prove, and we were not able to find a direct proof in the literature. We had some probabilistic arguments that affirmed the result for almost all mean payoff games. This alone was enough for us to use it as a metric for probabilistic strategies.
\newline Eventually, we were able to give a formal proof, which is detailed in the appendix \ref{appendix:Probabilistic:Strategies}.
\subsubsection{Game Value}
By combining both theorems \ref{theorem:OptimalStrategy} and \ref{theorem:MeanPayoffExistence}, there is a mean payoff $v(G)$ that both players are guaranteed to achieve if they play optimally:
\begin{align*}
\exists \Phi^{\Max},\forall \Phi^ {\Min}, & \quad v^-(G,\Phi^{\Max},\Phi^{\Min}) \ge v(G) \\
\exists \Phi^{\Min},\forall \Phi^{\Max}, & \quad v^+(G,\Phi^{\Max},\Phi^{\Min}) \le v(G) 
\end{align*}
Such mean payoff is called the value of the game. This value determines the winner of the game assuming both players play optimally.
\section{Evaluating Strategies}
\label{section:StrategyEvalution}
Suppose we have a pair of potentially probabilitic strategies $(\Phi^{\Max},\Phi^{\Min}).$ The problem is to evaluate the winner without doing an infinite simulation of the game. 
\subsection{Monte-Carlo Simulations}
This is the most intuitive evaluation method 
\subsection{Positional Strategies}
If both strategies are deterministic and memoryless. Then the generated sequence of vertices $(s_n)_{n\in\mathbb{N}}$ will be completely determined by the recurrence relation:
$$
s_n=\begin{cases}
	s & \text{if } \space n=0 \\
	\Phi^{\Max}(s_{n-1})& \text{if}\ n \ \text{is odd} \\
	\Phi^{\Min} (s_{n-1}) & \text{otherwise}
\end{cases}
$$
This can be represented in the compact form:
$$
\forall n\in\mathbb{N}^*,\quad \left(s_n, p_n
\right) = \left(\Phi^{p_{n-1}}(s_{n-1}), \bar{p}_{n-1}\right) = F(s_{n-1},p_{n-1})
$$

Since $\mathcal{V} \times \mathcal{P}$ is a finite set and $F$ is a function, such sequence will be eventually periodic, that is:
$$
\exists N \in \mathbb{N},\exists T\in\mathbb{N}^*/\quad \forall n\in\mathbb{N}_{\ge N},\quad (s_{n},p_{n})=(s_{n+T},p_{n+T})
$$

We can calculate its eventual period using the turtle hare algorithm. %Reference?

Now, the mean payoff will be equal to the mean of weights that appears on the cycle. \newline
This can be proven as follow.:
% w_n has offset N
\begin{align*}
	S_{aT+b+N}&=\sum_{k=0}^{aT+b+N-1} w_{k} \\
	&= \sum_{k=0}^{N-1}  w_{k}  + \sum_{k=0}^{aT+b-1} w_{k+N} \\
	&= \sum_{k=0}^{N-1}  w_{k}  + a\sum_{r=0}^{T-1} w_{r+N} +  \sum_{r=0}^{b-1} w_{r+N} \\
	\implies \left \lvert S_{n+N}- \lfloor \frac{n}{T}\rfloor \sum_{r=0}^T w_{k+N}  \right\rvert &\le (N+T-1) \max_{(u,v)\mathcal{E}} \lvert \mathcal{W}(u,v) \rvert  \\
	&\le (N+T-1) \lVert \mathcal{W} \rVert_{\infty} \\
\implies \left \lvert \frac{1}{n+N}S_{n+N}-\frac{1}{n+N} \cdot \lfloor \frac{n}{T}\rfloor \sum_{r=0}^{T-1} w_{k+N}  \right\rvert &\le \frac{N+T-1}{n+N}  \lVert \mathcal{W} \rVert_{\infty} \\
\end{align*}
Now it can be proven that:
$$
\lim_{n\rightarrow +\infty } \frac{1}{n+N} \cdot \lfloor \frac{n}{T}\rfloor \sum_{r=0}^T w_{k+N}  = \frac{1}{T}\sum_{r=0}^{T-1}w_{r+N}
$$
With that:
$$
\lim_{n\rightarrow +\infty} \frac{1}{n}\sum_{k=0}^{n-1} w_k=\frac{1}{T}\sum_{r=0}^{T-1}w_{r+N}  \quad \blacksquare
$$
Now, our algorithm will be composed of $3$ main parts:
\begin{itemize}
	\item Calculating the transition function $F:\VertexSet\times \PlayerSet \rightarrow \VertexSet\times \PlayerSet$. This is straightforward from the construction.
	\item Calculating the period and the offset of the sequence. We will use Floyd's cycle finding algorithm for that.
	\item Calculating the Mean Payoff
\end{itemize}
This is an illustrative implementation of our algorithm.
\begin{algorithm}
	\caption{Deterministic strategies evaluation}\label{alg:DeterministicEvaluation}
	\begin{algorithmic}
		\Require $G=(V,E,P,s,p)$ a mean payoff game
		\Require $(\Phi^{\text{Max}},\Phi^{\text{Min}})$ the edge probability 
		\Ensure $R$ The mean payoff  
		\State $F\leftarrow \text{Transition}(G,\Phi^{\text{Max}},\Phi^{\text{Min}})$\Comment{Calculate the transition function}

		\State $x_0\leftarrow (s,p)$
		\State $(T,r)\leftarrow \text{FloydCycleFinding}(F,x_0)$ \Comment{Find the period and the offset}
		\State $S\leftarrow 0$ \Comment{$S$ represents the cumulative payoffs along a cycle}
		\State $x\leftarrow x_0$
		\For {$k\in\{1,\dots,r\}$} \Comment{Advance until arriving to the cycle}
			\State $x\leftarrow F(x)$
		\EndFor
		\For {$k\in \{1,\dots,T\}$}
			\State $y\leftarrow \leftarrow F(x)$
			\State $u\leftarrow \displaystyle\projection_{V\times P\rightarrow V}(x)$ \Comment{Extracts the current vertex}
			\State $v\leftarrow \displaystyle\projection_{V\times P\rightarrow V}(y)$ \Comment{Extracts the next vertex}
			\State $S\leftarrow S+W(u,v)$
			\State $x\leftarrow y$
		\EndFor
		\State \Return $R\leftarrow \frac{S}{T}$
	\end{algorithmic}
\end{algorithm}
\FloatBarrier


\subsection{Probabilistic Strategies}
Due to the undeterministic nature of probabilistic strategies, it does not make sense to evaluate the mean payoffs, as different executions may lead to different mean payoffs. \newline
%Proof of discrete distribution nature
%My intuition says that it may not a distribution 
Instead, probabilistic strategies gives rise to a discrete distribution of mean payoffs. \newline
Now two closely related, but different evaluations are possible
\begin{itemize}
	\item Expected Mean Payoff
	\item Distribution of winners 
\end{itemize}
Now, with both strategies fixed. A Mean Payoff Game can be considered as a Markov Chain.

\begin{algorithm}
	\caption{Probabilistic strategies evaluation}\label{alg:ProbabilisticEvaluation}
	\begin{algorithmic}
		\Require $G=(V,E,P,s,p)$ a mean payoff game
		\Require $(\Pi^{\text{Max}},\Pi^{\text{Min}})$ the edge probability 
		\Ensure $\mathbb{E}[R]$ The expected mean payoff  
		\State $(A,W)\leftarrow \text{MRP}(G,\Phi^{\text{Max}},\Phi^{\text{Min}})$\Comment{Extract the MRP form. This is detailed in \ref{section:ProbabilisticStrategies:MRP}}
		
		\State $u\leftarrow (A\odot W)\ones$
		\State $X\leftarrow \text{NullSpace}(\Id - A)$ \Comment{Extract the kernel-basis of $\Id - A$}
		\State $Y\leftarrow \text{NullSpace}(\Id - A^T)$ \Comment{Extract the kernel-basis of $\Id - A^H.$}
		\State $T \leftarrow X(Y^TX)^{-1}Y^T$ \Comment{Calculate the limit $\displaystyle\lim_{n\rightarrow +\infty}\frac{1}{n}\sum_{k=0}^{n-1}A^k.$ This is detailed in \ref{section:ProbabilisticStrategies:AverageTimeReward}}
		\State \Return $\mathbb{E}[R]\leftarrow Tu$
	\end{algorithmic}
\end{algorithm}


\section{Countering Strategies}
By fixing the strategy of player $p\PlayerSet$ to $\Phi^p,$ then the Markov Game Process reduces to a Markov Decision Process, which can be solve by Linear Programming tools %Reference?
\newline Moreover, in a deterministic Mean Payoffs, a counter strategy can be calculated efficiently by reducing the problem to finding a negative cycle in a di-graph.
\subsection{Deterministic Counter Strategy}
For simplicity, we assume that $p=\Max,$ the same results apply for $\Min$ player.
\newline For a Mean Payoff Game $(\VertexSet,\EdgeSet,\PlayerSet,s,p'),$ we introduce the following graph:
$$
G'=\left(V,E',W'\right)
$$
where:
\begin{itemize}
	\item $E'$ is defined as follow:
	$$
	E'=\left\{(u,\Phi^p(v)),\quad (u,v)\in E\right\}
	$$
	\item Also, $W'$ is defined adequately:
	$$
	\forall (u,v)\in E: W'(u,\Phi^p(v))=W(u,v)+W(v,\Phi^p(v))
	$$
\end{itemize}	
The problem of finding a counter strategy will be reduced to finding a negative cycle $\mathcal{C}$ in $G'$
\newline This can be done in $\mathcal{O}(\lvert V\rvert^3)$
\subsection{Probabilistic Counter Strategy}
On the other hand, if $\Phi^p$ is probabilistic, then the problem can be reduced to optimizing the mean payoff of an infinite-horizon Markov Decision Process.

\section{Learning Strategies}
\section{State of the Art}
Mean Payoff Games are well-known in many fields, such that Optimization\cite{SimplexMPG}, Game Theory\cite{PositionalStrategies}, Formal Verification\cite{OmegaSpecsMPG}, Constraint Satisfaction Problems\cite{TropicalCSP,MPGMaxAtom}, Reinforcement Learning\cite{StrategyImprovement}.
\newline While we were not able to trace the exact origin of Mean Payoff Games, we were able to find references to it since the seventies \cite{PositionalStrategies}. The problem itself is interesting as it connect many related fields. First of all, it is closely related to many problems in constraint satisfaction \cite{TropicalCSP,MPGMaxAtom}, model-checking \cite{OmegaSpecsMPG}, game theory \cite{PositionalStrategies}.
\newline Also, another interesting fact is that deciding the winner of a mean payoff game is polynomial time equivalent\footnote{Each instance of both problems can be transformed to the latter in polynomial time.} to the Max Atom problem \cite{MPGMaxAtom}, which is in $\texttt{NP}\cap \texttt{co-NP},$ but its membership to $\mathtt{P}$ is still open. This is remarkable, there only few problems that share such fate \cite{NPInterCoNP}.


This influenced mainly two research axes. The first deals with solving the decision problem\footnote{The decision problem of a mean payoff game is deciding the winner.}, and also optimization problem\footnote{The optimization problem is calculating the best strategy for each player.} related to calculating the optimal strategies. 
\newline The optimization problem itself can be solved using exact methods \cite{MPGMaxAtom}, as well as iterative methods \cite{StrategyImprovement,SimplexMPG}.

While we did not find a machine learning approach on mean payoff games in the literature, we were able to find some results in a superclass, known as stochastic parity games. In fact, a model free reinforcement learning approach was proposed using the Q-learning minimax algorithm , as well as a supervised learning approach on solved instances of that game.

\section{Library Implementation}
We have two implementations of our mean payoff game library:
\begin{itemize}
	\item The first is called \textbf{mpg}. It is implemented in Python, and it contains the core functionalities of mean payoff games, as well as serialization, visualisation and support for machine learning methods.
	\item The second one is called \textbf{mpgcpp}. It is implemented in C++ to maximimize effeciency.
\end{itemize}
Also, the time-critical C++ functions are exported to Python. In they can be used via the \textbf{wrapper} module.
\subsection{mpg}
We have implemented a library called \textbf{mpg} that contains all the functionalities that we have discussed for mean payoff games.
\newline Here we list the modules of that library:
\begin{figure}[H]
	\centering
	    \begin{tikzpicture}
	    	
	   \begin{package}{ mpg }
		\begin{class}[text width=3cm]{csp}{0,0}
		\end{class}
		
		\begin{class}[text width=3cm]{games}{4,0}			
		\end{class}

		\begin{class}[text width=3cm]{graphs}{8,0}
		\end{class}
		
		\begin{class}[text width=3cm]{ml}{0,-2}
		\end{class}
		
		\begin{class}[text width=3cm]{rl}{4,-2}
		\end{class}
		
		\begin{class}[text width=3cm]{sp}{8,-2}
		\end{class}
		
		
		\begin{class}[text width=3cm]{gnn}{0,-4}
		\end{class}
		
		\begin{class}[text width=3cm]{visualisation}{4,-4}
		\end{class}
		
		\begin{class}[text width=3cm]{wrapper}{8,-4}
		\end{class}
		\end{package}
	\end{tikzpicture}
	\caption{\textbf{mpg} library}
\end{figure}
\FloatBarrier
This library contains the following modules:
\subsubsection{csp}
This module contains the constraint satisfaction methods used to solve Mean Payoff Games. 
\newline They are describe in details in the appendix \ref{appendix:CSP}.
\subsubsection{graphs}
This modules contains some graph algorithms needed for Mean Payoff Games, such as:
\begin{itemize}
	\item Floyd-Warshall method to find negative cycles.
	\item Methods to generate random graphs as described in section \ref{section:Dataset}. The theoretical details are in the appendix \ref{appendix:RandomGraphs}
\end{itemize}
\subsubsection{visualisation}
This module serves as a front-end for Jupyter Notebook, so we can visualise mean payoff games, and also the strategy of each player.
\begin{figure}[H]
	\includegraphics[width= \textwidth]{Figures/OptimalPlayNetworkx.png}
	\caption{Generated visualisation for a Mean Payoff Game with the optimal strategies}
\end{figure}
\FloatBarrier
\subsubsection{games}
This module defines the core functionalities related to mean payoff games.
\newline It also defines the methods to read/write Mean Payoff Graphs. It uses the weighted edgelist format, and supports compression.  
\begin{figure}[H]
	\centering
	\begin{tikzpicture}
		\small
		\begin{package} {networkx}
			\begin{class}{DiGraph}{0,2.5cm}
			\attribute{\dots}
			\operation{\dots}
			\end{class}
		\end{package}
		\begin{package}{ games }
			\begin{class}[text width=12cm]{MeanPayoffGraph}{0,0}
				\inherit{DiGraph}
				\attribute {+ weights\_matrix: Matrix}
				\attribute {+ adjacency\_matrix: Matrix}
				\attribute {+ tensor\_representation: Tensor[3]}
				\operation{+ closure() : MeanPayoffGraph}
				\operation{+ dual() : MeanPayoffGraph}
				\operation {+ as\_bipartite() : MeanPayoffGraph}
				\operation {+ as\_min\_max\_system() : MinMaxSystem}
			\end{class}
			
			\begin{interface}[text width=8.5cm]{Strategy}{0,-5cm}
				\operation{+ \_\_init\_\_(G : MeanPayoffGraph, player : bool)}
				\operation[0]{+ call(vertex : int) : int}
			\end{interface}
			
			\begin{class}[text width=3.5cm]{PositionalStrategy}{-2cm,-8cm}
				\inherit{Strategy}
				\operation[0]{+ call(vertex)}
			\end{class}
			
			\begin{class}[text width=3.5cm]{GreedyStrategy}{-6cm,-8cm}
								\inherit{Strategy}
				\operation[0]{+ call(vertex)}
			\end{class}
			\begin{class}[text width=3.5cm]{EpsGreedyStrategy}{2cm,-8cm}
								\inherit{Strategy}
				\operation[0]{+ call(vertex)}
			\end{class}
			
			\begin{class}[text width=3.5cm]{FractionalStrategy}{6cm,-8cm}
								\inherit{Strategy}
				\operation[0]{+ call(vertex)}
			\end{class}
			
			\begin{class}[text width= 12cm]{Functions}{0,-10cm}
				\operation {@ winning\_everywhere() : bool}
				\operation {@ winning\_somewhere() : bool}
				\operation{@ mpg\_from\_digraph(G : nx.Digraph) : MeanPayoffGraph}
				\operation{@ optimal\_strategy\_pair(G : MeanPayoffGraph) : Tuple[Strategy,Strategy]}
				\operation{@ counter\_strategy(G : MeanPayoffGraph, psi: PositionalStrategy, source : int, player : bool) : Strategy}
				\operation{@ mean\_payoff(G: MeanPayoffGraph, source: int, psi1 : PositionalStrategy, psi2 : PositionalStrategy, turn : bool) : float}
				\operation{@ mean\_payoffs(G: MeanPayoffGraph, psi1 : PositionalStrategy, psi2 : PositionalStrategy) : Matrix}
				\operation{@ winner(G: MeanPayoffGraph, source: int, psi1 : PositionalStrategy, psi2 : PositionalStrategy, turn : bool) : bool}
				\operation{@ winners(G: MeanPayoffGraph, psi1 : PositionalStrategy, psi2 : PositionalStrategy) : Matrix}
			\end{class}
		
		\end{package}
	\end{tikzpicture}
	\caption{\textbf{games} module}
\end{figure}
\FloatBarrier
\subsubsection{ml}
This module defines the required layers, blocks, and model architectures to do machine learning on Mean Payoff Games. This is detailed in chapter \ref{section:ModelDesign}.
\subsubsection{gnn}
This module defines the basic functionalities of graph neural networks \cite{GNN} that are required for our models.

\subsubsection{rl}
This module defines the required functions to do reinforcement learning on Mean Payoff Games\footnote{The current version of this module only supports Mean Payoff Games where at least one player has a fixed strategy.}.

\subsubsection{sp}
This module defines a basic AlphaZero based agent to learn the game.
\subsubsection{wrapper}
This module contains a binding to the C++ implementation of time-critical methods for mean payoff games.
\newline We will give details about this wrapper in the next section.
\subsection{mpgcpp}
This library contains the 
\begin{figure}[H]
	\centering
	\begin{tikzpicture}
		
		\begin{package}{ mpg }
			\begin{class}[text width=3cm]{csp}{0,0}
			\end{class}
			
			\begin{class}[text width=3cm]{games}{4,0}			
			\end{class}
			
			\begin{class}[text width=3cm]{mpgio}{8,0}
			\end{class}
		\end{package}
	\end{tikzpicture}
	\caption{\textbf{mpgcpp} library}
\end{figure}
\subsubsection{csp}
This module contains the constraint satisfaction methods used to solve Mean Payoff Games. 
\subsubsection{games}
This module defines the core functionalities related to mean payoff games.
\begin{figure}[H]
	\centering
	\begin{tikzpicture}
		\small
		\begin{package}{games}
			\begin{abstractclass}[text width=6cm]{MeanPayoffGraphBase}{0,5.5cm}
			\attribute{edges}
			\attribute{dual\_graph}
			\operation[0]{\# add\_edge\_impl(source, target, weight)}
			\operation[0]{\# set\_dual(dual)}
			\operation[0]{+ get\_weight(src, dest)}
			\operation{+ get\_weights()}
			\operation{+ count\_nodes() : int}
			\operation{+ get\_edges() : Edges}
			\end{abstractclass}
			\begin{class}[text width=3.5cm]{VectorMPG}{-6cm,4cm}
				\inherit{MeanPayoffGraphBase}
			\end{class}
			
			\begin{class}[text width=3.5cm]{MatrixMPG}{-6cm,0}
				\inherit{MeanPayoffGraphBase}
			\end{class}
			
			\begin{class}[text width=3.5cm]{HashMapMPG}{6cm,4cm}
				\inherit{MeanPayoffGraphBase}
			\end{class}
			
			\begin{class}[text width=3.5cm]{TreeMapMPG}{6cm,0}
				\inherit{MeanPayoffGraphBase}
			\end{class}
			
			\begin{class}[text width=3.5cm]{DualMPG}{0cm,0cm}
				\inherit{MeanPayoffGraphBase}
			\end{class}
			
			
			\draw[umlcd style,-diamond,fill opacity=100] (DualMPG) to [bend right=12]  (MeanPayoffGraphBase); 
			%\draw[umlcd style,triangle,fill opacity=100] (DualMPG) to [bend right=-12]  (MeanPayoffGraphBase); 
		
			
			\begin{interface}[text width=7.5cm]{Strategy}{0,-2cm}
				\operation{+ \_\_init\_\_(G : MeanPayoffGraph, player : bool)}
				\operation[0]{+ call(vertex : int) : int}
			\end{interface}
			
			\begin{class}[text width=3.5cm]{PositionalStrategy}{-6cm,-5cm}
				\inherit{Strategy}
				\operation[0]{+ call(vertex)}
			\end{class}
			
			\begin{class}[text width=3.5cm]{GreedyStrategy}{-2cm,-5cm}
				\inherit{Strategy}
				\operation[0]{+ call(vertex)}
			\end{class}
			\begin{class}[text width=3.5cm]{EpsGreedyStrategy}{2cm,-5cm}
				\inherit{Strategy}
				\operation[0]{+ call(vertex)}
			\end{class}
			
			\begin{class}[text width=3.5cm]{FractionalStrategy}{6cm,-5cm}
				\inherit{Strategy}
				\operation[0]{+ call(vertex)}
			\end{class}
			
			\begin{class}[text width= 12cm]{Functions}{0,-7cm}
				\operation {@ winning\_everywhere() : bool}
				\operation {@ winning\_somewhere() : bool}
				\operation{@ mpg\_from\_digraph(G : nx.Digraph) : MeanPayoffGraph}
				\operation{@ optimal\_strategy\_pair(G : MeanPayoffGraph) : Tuple[Strategy,Strategy]}
				\operation{@ counter\_strategy(G : MeanPayoffGraph, psi: PositionalStrategy, source : int, player : bool) : Strategy}
				\operation{@ mean\_payoff(G: MeanPayoffGraph, source: int, psi1 : PositionalStrategy, psi2 : PositionalStrategy, turn : bool) : float}
				\operation{@ mean\_payoffs(G: MeanPayoffGraph, psi1 : PositionalStrategy, psi2 : PositionalStrategy) : Matrix}
				\operation{@ winner(G: MeanPayoffGraph, source: int, psi1 : PositionalStrategy, psi2 : PositionalStrategy, turn : bool) : bool}
				\operation{@ winners(G: MeanPayoffGraph, psi1 : PositionalStrategy, psi2 : PositionalStrategy) : Matrix}
			\end{class}
			
		\end{package}
	\end{tikzpicture}
	\caption{\textbf{games} module}
\end{figure}

\subsubsection{mpgio}
Unlike \textbf{mpg}, which has support for mean payoff graph I/O from files via \textbf{NetworkX}. In the C++ library, we have to implement this functionalities from scratch. We used standard I/O utilities with \textbf{boost} to interact with compressed streams conforming to the format used by \textbf{mpg}.
\subsection{Environment}
\subsection{Testing}
\subsection{Structure}

\chapter{Dataset Generation}

\section{Introduction}
\section{Analysis}
Generating a Mean Payoff Game can be decomposed into two subsequent objectives.
\begin{enumerate}
	\item Generate the Graph itself.
	\item Generate the Weights
\end{enumerate}


\section{Graph Distributions}
There are many well studied graph distributions in the litterature. \newline
One of the most explored ones are the $\mathcal{G}(n,p)$ and $\mathcal{G}(n,m)$ families.
\subsection{$\mathcal{G}(n,p)$ Family}
For $n\in\mathbb{N},p\in[0,1],$ a graph $G$ is said to follow a $\mathcal{G}(n,p)$ distribution if $\lvert V \rvert=n$ and:
$$
\forall e\in \mathscr{E}, \quad \mathscr{P}(s\in \mathcal{E})=p
$$
Where $\mathscr{E}$ is a set of valid edges. 

\subsection{$\mathcal{G}(n,m)$ Family}
For $n\in\mathbb{N},m\in\mathbb{N},$ a graph $G$ is said to follow a $\mathcal{G}(n,m)$ distribution if $\lvert V \rvert=n,\lvert  E \rvert=m$ and the edges $e_1,\dots,e_m$ were drawn from a set of valid edges $\mathscr{E}.$
\subsection{Valid edges}
The set of valid edges $\mathscr{E}$ is the set defining the potential edges of the graph. It is equal to:
\begin{enumerate}
	\item $V\times V$ for directed graphs with loops 
	\item $(V\times V)\setminus V\odot V$ for directed graphs without loops
	\item The set of subsets of size $2$ of $V$ denoted $\mathscr{P}_2(V)$ for undirected graphs with loops.
	\item The set of subsets of size $2$ of $V$ denoted $\mathscr{P}_2(V)$ for undirected graphs with loops.
\end{enumerate}
\subsection{$\mathcal{D}(n,p)$ Graph Construction}

\subsubsection{Naive Method}
The definition of $\mathcal{D}(n,p)$ gives a straightforward construction. \newline
This is achieved by flipping a coin\footnote{The coin is potentially biased with a probability of obtaining head equal to $p\in [0,1]$} for each pair of node $(u,v)\in V^2$, we add an edge if we get a Head. 
\newline This is implemented in the following algorithm:
\begin{algorithm}
	\caption{$\mathcal{D}(n,p)$ Graph Generation}\label{alg:Dnp_Naive}
	\begin{algorithmic}
		\Require $n\in\mathbb{N}^*$ the size of the graph
		\Require $p\in\mathbb{N}^*$ the edge probability 
		\Ensure $G\sim \mathcal{D}(n,p)$  
		\State $A:(u,v)\in V\times V\rightarrow 0$
		\For{ $u\in V$} 
		\For { $v \in V$ }
		\State Generate $X\sim \mathcal{B}(p)$
		\Comment{$\mathcal{B}(p)$ is the bernoulli distribution}
		\State $A(u,v)\leftarrow X$
		\EndFor
		\EndFor
		\State \Return $G\leftarrow \texttt{GraphFromAdjacencyMatrix}(A)$
	\end{algorithmic}
\end{algorithm}
\FloatBarrier
The complexity\footnote{We assume the cost of generating a Bernoulli random variable as $\mathcal{O}(1)$} of the following algorithm is $\mathcal{O}(n^2).$
\subsubsection{Optimized Method}
Instead of iterating over all possible pair of nodes. For each vertex $v\in V$:
\begin{itemize}
	\item We can sample a number $d$ from the outgoing degree distribution\footnote{Or the ingoing degree distribution, they are in fact equal.}
	\item We then choose $d$ numbers uniformly without replacement from an indexable representation of $V$
\end{itemize}
The following algorithm implements the optimized method:
\begin{algorithm}
	\caption{$\mathcal{D}(n,p)$ Graph Generation Optimisation}\label{alg:Dnp_Fast}
	\begin{algorithmic}
		\Require $n\in\mathbb{N}^*$ the size of the graph
		\Require $p\in\mathbb{N}^*$ the edge probability 
		\Ensure $G\sim \mathcal{D}(n,p)$  
		\State $A:u\in V\rightarrow \varnothing$
		\For{ $u\in V$} 
		\State Generate $d\sim \mathcal{B}(n,p)$
		\Comment{$d$ represents the degree, $\mathcal{B}(n,p)$ is the binomial distribution}
		\State $A(u)\leftarrow \choice(V,d)$
		\EndFor
		\State \Return $G\leftarrow \texttt{GraphFromAdjacencyList}(A)$
	\end{algorithmic}
\end{algorithm}
\FloatBarrier
Now, Let $C(n,m)$ be the cost of choice function. The expected complexity of this algorithm will be:
$$
\tilde{\mathcal{O}}\left(n\mathbb{E}_d[C(n,d)]\right) \quad \text{where}\ d\sim \mathcal{B}(n,p)
$$
We will show on the next section what choice function should we use.
\subsection{Choice Function}
\subsubsection{First Proposition}
We propose here a simple choice algorithm, but it is still efficient for our use case.
\newline It works simply by drawing without replacement, but we ignore duplicate elements. This is implemented as follow
\begin{algorithm}
	\caption{$\mathcal{D}(n,p)$ Choice without replacement}\label{alg:Choice}
	\begin{algorithmic}
		\Require $S$ a list
		\Require $m\in\{0,\dots \lvert S \rvert\}$ the number of chosen elements
		\Ensure $H$ a set of size $m$ containing uniformly drawn elements without replacement. 
		\State $H\leftarrow \varnothing$
		\While{$\lvert H \rvert < m$}
			\State Generate $v\sim \mathcal{U}(S)$ 
			\Comment{Where $\mathcal{U}(S)$ is the uniform distribution over $S$}
			\State $H\leftarrow H \cup \{v\}$
		\EndWhile
		\State \Return $H$
	\end{algorithmic}
\end{algorithm}
\FloatBarrier
To estimate the cost of this algorithm, we will use probabilistic reasoning.
\newline Let $X_{n,m}=C(n,m)$ the running time of an execution of algorithm \ref{alg:Choice} in a set $S$ of size $n$, with $m$ elements to be chosen.
We have:
\begin{align*}
	X_{n,0} & \ \text{is deterministic}\\
	X_{n,0}&=\mathcal{O}(1) \\
	\mathbb{E}[X_{n,m}]&=1+\frac{1}{n}\sum_{k=0}^{n-1} \mathbb{E}[X_{n,m} \mid \text{The last drawn number is}\ k] \\
	&=1+\frac{1}{n}\sum_{k=0}^{m-2} \mathbb{E}[X_{n,m}]+\frac{1}{n}\sum_{k=m-1}^{n-1} \mathbb{E}[X_{n,m-1}] \\
	&= 1+\frac{m-1}{n}\mathbb{E}[X_{n,m}]+\frac{n-m+1}{n}\mathbb{E}[X_{n,m-1}]
\end{align*}
Now we arrived at a recurrent formula. We will simplify it as shown below:
\begin{align*}
\frac{n-m+1}{n}\mathbb{E}[X_{n,m}]&=\frac{n-m+1}{n}\mathbb{E}[X_{n,m-1}] +1\\
\implies \mathbb{E}[X_{n,m}]&=\frac{n-m+1}{n-m+1}\mathbb{E}[X_{n,m-1}]+\frac{n}{n-m+1}\\
&=\mathbb{E}[X_{n,m-1}]+\frac{n}{n-m+1} \\
&=\sum_{k=1}^m\frac{n}{n-k+1}+\mathcal{O}(1)\\
&=\sum_{k=0}^{m-1}\frac{n}{n-k}+\mathcal{O}(1) \\
&=n\sum_{k=n-m+1}^n\frac{1}{k}+\mathcal{O}(1)\\
&=n(H_n-H_{n-m})+\mathcal{O}(1)
\end{align*}
Here $(H)_{n\in\mathbb{N}^*}$ is the harmonic series, and we define $H_0=0.$
\subsubsection{Complexity}
%Prooof!!!
The expected complexity of algorithm \ref{alg:Choice} depends on both $n$ and $m$:
\begin{itemize}
	\item If $m=o(n),$ then it is $\tilde{\mathcal{O}}(m).$
	\item If $m=kn+o(n)$ with $k\in\mathopen]0,1\mathclose[$, then it is $\tilde{\mathcal{O}}(m).$
	\item If $m=n-o(n)$, It is\footnote{Here we use the minus sign to emphasize that $m\le n$} $\tilde{\mathcal{O}}(m\log m).$ 
\end{itemize}
To prove this result, we use a well known asymptotic approximation of the Harmonic series\footnote{This asymptotic approximation can be proven using the Euler–Maclaurin formula}:
$$
H_n=\ln n+\gamma -\frac{1}{2n}+\mathcal{O}\left(\frac{1}{n^2}\right)
$$
We can prove this claim as follow:
\begin{align*}
	m=o(n),\quad \mathbb{E}[C(n,m)]&=-n\ln \left(1-\frac{m}{n}\right) -\frac{1}{2}\left(1-\frac{n}{n-m}\right)+\mathcal{O}\left(\frac{1}{n}\right) \\
	&=m+o(m)\\
	&=\mathcal{O}(m)\\
	m=km+o(m),k\in \mathopen]0,1\mathclose[,\quad \mathbb{E}[C(n,m)]&=-n\ln \left(1-\frac{m}{n}\right) -\frac{1}{2}\left(1-\frac{n}{n-m}\right)+\mathcal{O}\left(\frac{1}{n}\right) \\
&=-n\ln (1-k+o(1))+\frac{1}{n}(1-\tfrac{1}{1-k+o(1)})+\mathcal{O}(\tfrac{1}{n})\\
&=\mathcal{O}(m)\\
\end{align*}
For $m=n-o(n),$ we prove it by noting that:
\begin{align*}
\mathbb{E}[C(n,m)]&\le\mathbb{E}[C(n,n)] \\
 &\le nH_n \\
&\le n\ln n +\gamma n +\frac{1}{2}+\mathcal{O}\left(\frac{1}{n}\right) \\
&= \mathcal{O}(m\log m)
\end{align*}

\subsubsection{Refinement}
If $m$ tends to $n,$ it is more hard to select $m$ elements from a set of size $n$ without replacement. This explains the extra logarithmic factor.
\newline In that case, we can instead focus on the dual problem: ``Find the $n-m$ elements that will not be selected". This can be calculated in $\mathcal{O}(n-m).$
\newline Once we find the elements that will not be selected, their set complement are exactly the $m$ elements that will be selected. This new algorithm is guaranteed to be $\mathcal{O}(m)$ irrespective of $n$ and $m$
\begin{algorithm}
	\caption{Fine tuned $\mathcal{D}(n,p)$ Choice without replacement }\label{alg:ChoiceFineTuned}
	\begin{algorithmic}
		\Require $S$ a list
		\Require $m\in\{0,\dots \lvert S \rvert\}$ the number of chosen elements
		\Require $\choice$ The choice function defined on algorithm \ref{alg:Choice}
		\Require $\tau$ a fine tuned threshold. We will use $\tau=\frac{1}{2}$ for all practical purposes.
		\Ensure $H$ a set of size $m$ containing uniformly drawn elements without replacement. 
		\If {$\frac{m}{\lvert S \rvert} \le \tau$}
			\State $H\leftarrow \choice(V,n)$
		\Else
			\State $H\leftarrow S\setminus \choice(S,n-m)$
		\EndIf
		\State \Return $H$
	\end{algorithmic}
\end{algorithm}
\FloatBarrier
Also, an important point is that by combining the analysis of all possible cases, we can extract a constant factor that is independent of $n.$ So that the Big-O notation is only a function of $m$.
\subsection{Complexity of Optimised $\mathcal{D}(n,p)$ Graph Construction}
We return to evaluate the asymptotic behaviour of $\mathbb{E}_d[C(n,d)].$
\newline Let $\delta \in \mathbb{R}_+^*$
\newline The Chebychev Inequality implies that:
$$
\mathscr{P}\left(\left \lvert \frac{1}{n}\sum_{i=1}^nX_i -p \right \rvert \ge \frac{\delta}{\sqrt{n}}\sqrt{p(1-p)}  \right) \le \frac{1}{\delta^2}
$$
By setting: $\delta=\frac{1}{\sqrt{p}},$ we have:
$$\mathscr{P}\left(\left \lvert \frac{1}{n}\sum_{i=1}^nX_i -p \right \rvert \ge \frac{\sqrt{1-p}}{\sqrt{n}}  \right) \le p 
$$
Let $I=[np-\sqrt{n(1-p)},np+\sqrt{n(1-p)}].$
\newline 
We have:
\begin{align*}
\mathbb{E}\left[C(n,d)\right]  &\le \mathbb{E}\left[C(n,d) \mid d\in I\right] + p^2\mathbb{E}\left[C(n,d) \mid d\notin I\right] \\
&\le \max_{m\in \mathbb{N} \cap I}C(n,m) + \max_{m\in\{0,\dots,n\}}p^2C(n,m) 
 \end{align*}
To further simplify this, we need two key observations:
\begin{itemize}
	\item The interval $m\in \mathbb{N} \cap [np-1+p,np+1-p]$ contains at most $3$ integers, all of which are within a distance of $1$ to $np$
	\item The complexity of $\mathbb{E}[C(n,m)]$ does only depend on $m.$
\end{itemize}
Thus, we have the following:
\begin{align*}
	\max_{m\in \mathbb{N} \cap [np-1+p,np+1-p]}C(n,m) &= \mathcal{O}(np) \\
	\max_{m\in\{0,\dots,n\}}p^2C(n,m) &= \mathcal{O}(np^2)
\end{align*}
Now, by combining both estimations, we get:
$$
\tilde{\mathcal{O}}(n\mathbb{E}_d[C(n,d)])=\tilde{\mathcal{O}}(n^2p) \quad \blacksquare
$$

\subsection{$\mathcal{D}(n,m)$ Construction}
To construct a random $\mathcal{D}(n,m)$ graph, we only have to select $m$ uniformly random elements from the set $V\times V.$
\newline We will use algorithm \ref{alg:ChoiceFineTuned} for this purpose\footnote{It is essential that the list $V\times V$ be lazy loaded. In particular, each element will only be loaded when it is indexed. This is essential to reduce the complexity. Otherwise, we will be stuck in an $\mathcal{O}(n^2)$ algorithm.}:
\begin{algorithm}
	\caption{Fine tuned $\mathcal{D}(n,p)$ Choice without replacement }\label{Dnm} 
	\begin{algorithmic}
		\Require $n\in\mathbb{N}^*$
		\Require $m\in\{0,\dots,n^2\}$ the number of chosen elements
		\Ensure $G\sim \mathcal{D}(n,m)$
		\State $E\leftarrow \choice(\text{Lazy}(V)\times \text{Lazy}(V),m)$ \Comment{We only need the $m$ elements on-demand.}
		\State \Return $G\leftarrow \text{GraphFromEdges}(E)$\Comment{This justifies using \text{Lazy}}
	\end{algorithmic}
\end{algorithm}
\FloatBarrier
Here $\text{Lazy}(V)\times \text{Lazy}(V)$ is a lazy implementation of cartesian product that supports bijective indexing\footnote{Indexing is required for uniform sampling} over $\{0,\dots,n^2-1\}.$
\newline The complexity of this construction is: $
\tilde{\mathcal{O}}(m)
$ 

\section{Sinkless Conditionning}
Sampling from a graph distribution may lead to graphs that have at least one sink. 
\newline These graphs are problematic as Mean Payoff Graphs are exactly the sinkless graphs.
\newline To migitate this, we will impose a conditionning on both distribution that will gives a guaranteed Mean Payoff Graph.
\newline We will explore such conditionning both distribution:
\begin{itemize}
	\item $\mathcal{G}^S(n,p):$ This is the distribution of graphs following $\mathcal{G}(n,p)$ with the requirement that they do not have a sink.
	\item $\mathcal{G}^S(n,m):$ This is the distribution of graphs following $\mathcal{G}(n,m)$ with the requirement that they do not have a sink.
\end{itemize}
\subsection{Repeating Construction}
\subsubsection{Algorithm}
This method is very intuitive. It will repeat the sampling until getting the desired graph. \newline The following is an implemention of the repeating construction.
\begin{algorithm}
	\caption{Fine tuned $\mathcal{D}(n,p)$ Choice without replacement }\label{alg:RepeatingConstruction}
	\begin{algorithmic}
		\Require $n\in\mathbb{N}^*$
		\Require $m\in\{0,\dots \lvert S \rvert\}$ the number of chosen elements
		\Require $\choice$ The choice function defined on algorithm \ref{alg:Choice}
		\Require Threshold $\tau$
		\Ensure $H$ a set of size $m$ containing uniformly drawn elements without replacement. 
		\If {$\frac{m}{\lvert S \rvert} \le \tau$}
		\State $H\leftarrow \choice(V,n)$
		\Else
		\State $H\leftarrow V\setminus \choice(S,n-m)$
		\EndIf
		\State \Return $H$
	\end{algorithmic}
\end{algorithm}

\subsubsection{Analysis}
We will analyse the runtime of generating a $\mathcal{G}^S(n,p).$
\newline We expect a similar runtime for $\mathcal{G}^S(n,m)$ due to the similarity between $\mathcal{G}(n,m)$ and $\mathcal{G}(n,p).$ 
\newline Let $F(n)$


\section{Weights Distribution}
\subsection{Construction}
Once the graph is constructed. We only have to generate the weights. \newline
This will be done by creating a random weight function:
$$
W(u,v):(u,v)\rightarrow W_{u,v}
$$
Here $W_{u,v}$ will be a sequence of real random variables. \newline
In our case, we set $(W_{u,v})_{(u,v)\in E}$ to be independent and identically distributed over a real distribution $\mathcal{W}.$ 


\section{Proposed MPG Distribution}
\subsection{Desired Properties of Mean Payoff Game Distributions}
\subsubsection{Fairness in the Limit}
This is essential, as we intend to generate a sequence of Mean Payoff Games that do not favour statistically a certain player, in the sense that, if we generate sufficient independent and identically distributed Mean Payoff Games $G_1,\dots,G_n$, we expect the following:
$$
\lim_{n\rightarrow +\infty} \left\lvert \mathtt{R}_{\Max}(G_1,\dots,G_n)-\mathtt{R}_{\Min}(G_1,\dots,G_n)\right \rvert = 0
$$
Where $\mathtt{R}$ is defined as follow:
$$
\mathtt{R}_{\text{Op}}(G_1,\dots,G_n)=\frac{1}{n}\sum_{i=1}^n\mathscr{P}(\text{Op wins} \ G_i\ \text{assuming optimal strategies})
$$
\subsubsection{Symmetric}
A real distribution is said to be symmetric if:
$$
\forall [a,b]\in \mathbb{R},X\sim \mathcal{W},\quad \mathscr{P}(X\in [a,b]) = \mathscr{P}(X\in [-b,-a])
$$
We will define a symmetric Mean Payoff Game distribution as a distribution of Mean Payoff Game whose weights are independent and identically distributed on a symmetric real distribution.
This property is stronger than Fairness in the Limit, as it implies that:
$$
\mathscr{P}(\text{Max wins} \ G\ \text{assuming optimal strategies}) = \mathscr{P}(\text{Min wins} \ G\ \text{assuming optimal strategies})
$$
We will require a Symmetric Mean Payoff Game as we do not want a player to have an inherit advantage other the other one\footnote{Other than the first move.}
\subsection{Implemented Distributions}
The following table resumes the implemented distributions:
\begin{table}[h]
	\small
	\begin{tabularx}{\textwidth}{| X | X | X |}
		\hline
		
		Distribution Family & Parameters & Type  \\
		\hline
		$\mathcal{D}(n,p)$ & \vspace{-5mm}
		\begin{itemize}
			  \setlength\itemsep{0em}
			\item $n:$ Graph size
			\item $p:$ Edge probability
		\end{itemize} & Graph distribution \\
		\hline
		$\mathcal{D}(n,m)$ & 
		\vspace{-5mm}
		\begin{itemize}
			  \setlength\itemsep{0em}
			\item $n:$ Graph size
			\item $m:$ Number of edges
		\end{itemize} & Graph distrbiution  \\
		\hline
		$\mathcal{U}_{\text{discrete}}(-r,r)$ &
		\vspace{-5mm}
		\begin{itemize}
			  \setlength\itemsep{0em}
			\item $r:$ The radius of the support
		\end{itemize}
		 &  Weight distribution\\
		\hline
		$\mathcal{U}(-r,r)$ &\vspace{-5mm}
		\begin{itemize}
			  \setlength\itemsep{0em}
			\item $r:$ The radius of the support
		\end{itemize} & Weight distribution \\
		\hline
		$\mathcal{N}(0,\sigma)$ &
		\vspace{-5mm}
		\begin{itemize}
			  \setlength\itemsep{0em}
			\item $\sigma:$ The standard deviation
		\end{itemize} & Weight distribution\\ 
		\hline 
		
	\end{tabularx}
	\caption{Le tableau d'avancement des BNNs}
\end{table}
\FloatBarrier


\section{MPG Generation}
\subsection{Distribution}
\begin{itemize}
	\item Each generated graph will follow a distribution $\mathcal{G}(n,p(n))$  for some $n\in\mathbb{N}^*$
	\item The weights will follow the discrete uniform distribution $\mathcal{D}(-1000,1000)$

\end{itemize}

We will generate two kinds of datasets, depending on the nature of the graph

\subsection{Dense Graphs}
\begin{itemize}
	\item Let $\mathcal{P}=\{0.1,0.2,0.3,0.5,0.7,0.8,0.9,1\}$
	\item $\mathcal{N}=\{10,20,30,40,50,60,70,80,90,100,110,120,130,140,150,200,250,300,400,500\}$
	\item For each $(n,p)\in \mathcal{N}\times \mathcal{P},$ we will generate $K=1000$ observations $G^{n,p}_1,\dots,G^{n,p}_{K} \sim \mathcal{G}(n,p)$ 
\end{itemize} 

The total number of examples is:
$$
K\times\lvert \mathcal{N} \rvert \times \lvert \mathcal{P}\rvert=160000
$$
The generation was done on a `haswell64` partition with 24 cores. and it took 02:12:38 hours.

\section{Annotation}
\subsection{Approach}
We used the CSP algorithm $\ref{alg:AC3Optimized}$ to annotate the dataset, potentially augmented with some heuristics. 
\newline We implemented a program that takes the path of the dataset, and solves the Mean Payoff Games one by one.
\newline To maximize efficiency, the program launches many solver threads, with each one independently working on a single file, and the results are accumulated using a ConcurrentQueue.
\subsection{Target Values}
The solver will calculate the following targets:
\begin{itemize}
	\item The optimal pair of strategies
	\item The mean payoffs for each starting position, turn.
	\item The winners for each starting position, turn.
\end{itemize}
Also, some additional metadata are generated for analysis:
\begin{itemize}
	\item \texttt{dataset}: The name of the whole dataset
	\item \texttt{graph}: The name of the graph.
	\item \texttt{status}: The solver's status on the given graph. In particular, whether it succeeded to solve the instance or not\footnote{We expect that the solver may crash due to several reasons (corrupted file, out of memory, etc\dots). For that we made additional effort for exception handling, so that an error for a single instance does not propagate to the whole program.}. Equal to ``OK" if the execution is successful.
	\item \texttt{running\_time}: The time needed to solve the instance.

\end{itemize}

\subsection{Heuristics}
To accelerate the annotation of the two datasets, we had to apply some heuristics to the algorithm. We made essentially two kinds of heuristics.
\subsubsection{Linear Bound Heuristic}
This is the heuristic based on the view that for almost all solutions of a Ternary Max Atom system extracted from our generated random games, either:
\begin{itemize}
	\item All variables are infinite: 
	$$
	X(u)=-\infty \quad \forall u\in V$$
	\item The diameter of assignments is in the order of $\lVert W\rVert_{\infty}$
	$$
	\Delta X = \sup_{u\in V}X(u)-\inf_{u\in V,X(u)>-\infty}X(u)= \mathcal{O}(\lVert W\rVert_{\infty})$$
\end{itemize}
This heuristic suggests a much tighter search space to the worst case $\lVert W \rVert_{1}$ one. Going further, with uniform random weights:
$$
\lVert W \rVert_{1}=\mathcal{O}\left(\lvert E \rvert \times \lVert W \rVert_{\infty}\right)
$$
We believe this heuristic arises due to the random property of graphs, because in general, one can build an infinite family of ternary max atom systems that violate this heuristic. \newline In fact, going further, one can build ternary max atom systems were the $\lVert W \rVert_{1}$ estimation is tight.
%Example of such system:
% Linear chain of constraints 

To generating the dataset, we applied this heuristic with $\Delta X=4\lVert W\rVert_{\infty}$
$$
D = \{-\infty,-2 \lVert W\rVert_{\infty},,-2 \lVert W\rVert_{\infty}+1 ,\dots ,2\lVert W\rVert_{\infty} \}
$$

\begin{figure}[H]
	\centering
	\begin{tikzpicture}[node distance={20mm}, thick, main/.style = {draw, circle}]
		\node[main] (1) {$0$}; 
		\node[main] (2) [right of =1] {$1$}; 
		\node[main] (3) [right of =2] {$2$}; 
		\node[main] (4) [right of =3] {$3$}; 
		\node[main] (5) [right of =4] {$\dots$}; 
		\node[main] (6) [right of =5] {$n-1$}; 
		\node[main] (7) [right of =6] {$n$}; 
		\draw[->] (1) -- node[midway, above right, sloped, pos=0.3] {-5} (2);
		\draw[->] (2) -- node[midway, above, sloped, pos=0.5]{6} (3);
		\draw[->] (3) -- node[midway, above, sloped, pos=0.5] {-8} (4);
		\draw[->] (4) -- node[midway, above, sloped, pos=0.5] {-5}(5);
		\draw[->] (5) -- node[midway, above, sloped, pos=0.5] {7}(6);
		\draw[->] (6) -- node[midway, above, sloped, pos=0.5] {2}(7);
	\end{tikzpicture} 
	\caption{A counter example of the Linear Bound heuristic}
\end{figure}

\subsubsection{Early Stopping}
If after any iteration of arc consistency, $\max_{x\in V}\nu(x) < \sup D.$ Then, $\nu(t)$ will converge to $-\infty$ for all $t.$ 
\newline Thus, we stop the algorithm and sets $\nu(t)\leftarrow -\infty,\quad \forall t$
\paragraph{Proof} suppose that in fact there is an assignment with: 
$$
-\infty < \max_{u\in V}\nu(u) < \sup D$$
We will take the $u$ with the biggest such $\nu(u).$  
\newline Now our system is a tropical max atom system, which means translations are also a polymorphism of this system, so for any assignment $\nu:V\rightarrow\mathbb{Z},\ X+t$ is also an assignment $\forall t\in \mathbb{Z}.$ With that, $\nu+\sup D-\nu(u)$ is also an assignment.
\newline This assignment has the property: $$
\forall s\in V,\quad \nu(s)+\sup D-\nu(u) \in D
$$
Which is a contradiction, as it violates the consistency of arc consistency, and the maximality of the solution with respect to the domain $D$

  
The efficiency of the Early Stopping heuristic depends on the density of the graph. Empirically, for dense graphs. the analoguous ternary max atom system has two kind of assignments:
\begin{enumerate}
	\item Either all variables are finite
	\item Either all variables are $-\infty$
\end{enumerate}
This translates back in a dense Mean Payoff setting, that the winner of the game does not depend in the starting position and the starting turn. 
\newline With that, the Early Stopping heuristic will quickly detect the second case, which is the usually hurdle of the algorithm.
\newline On the other hand, for sparse graphs, we do not have this nice distinction between finite and infinite assignments, and they can overlap, and so will make this heuristic useless in practice.

\subsection{Deployment}
After some experiments, it was very clear that vertical scaling with the number of threads is not sufficient. By analysing the running time of some samples, we estimated the total running time solving both datasets to exceed $30$ days.
\newline As a result of this, we deployed a pipeline of $24$ nodes, each with $24$ threadss working simultaneously on a partition of the dataset.
 
\chapter{Model Design}

\section{Introduction}

\chapter{Reinforcement learning and Self play approach}
\section*{Introduction}

In this final chapter, we will define \acrfull{rl} more formally, and then briefly describe the difference of some \acrshort{rl} approaches.

Once we have the \acrshort{rl} background, we will transform our \acrshort{mpg} into a \acrfull{sg} in a consistent way with the model architecture defined on chapter \ref{chapter:ModelDesign}.

Then, we will begin on \acrfull{sp} system. First, we will describe vanilla \acrshort{mcts} implementations with some required modifications to the \acrshort{sg} for convergence, and later, we will discuss the \acrshort{mcts} based on Alpha Zero that uses the model defined on chapter \ref{chapter:ModelDesign}.

Later, we will design and implement the \acrshort{sp} pipeline, deploy it to a \acrshort{hpc} pipeline.

Finally, we will create a configuration file for externalizing the configuration, and then do some experiments with out \acrshort{sp} system 
\section{Reinforcement Learning}
\subsection{Definition}
\citeauthor{RLIntroduction} \cite[Chapter.~1]{RLIntroduction} gave an excellent in-depth definition, which not only defines what \acrfull{rl} is, but also compare it with other \acrfull{ml} methods.
\newline We will summarize it as follow:
\begin{definition}
	
\end{definition}
\begin{figure}
	\centering
	\includegraphics[width= 0.8\textwidth]{Figures/RLDiagram.png}
	\caption{A Reinforcement Learning system}
\end{figure}
\subsection{Theory}
\subsubsection{Single Agent}
In most settings, the theory of \acrshort{rl} deals with environments where we would like to optimize a single agent. This is called single agent \acrshort{rl}. This is usually modeled within a \acrfull{mdp}
\parencites[chapter.~3]{RLIntroduction}[chapter.]~23]{AIModernApproach}

Now, the sole problem is the agent itself may or maynot know the model\footnote{In this particular context, a model of the \acrshort{mdp} is the representation of all transitions, their rewards, and the transition probabilities. That is the complete knowledge of the \acrshort{mdp}. This has nothing to do with our meaning of model.} of the \acrshort{mdp}. Thus there are mainly two approaches:
\begin{itemize}
	\item \textbf{Model-based \acrshort{rl}}.
	\item \textbf{Model-free \acrshort{rl}}. 
\end{itemize}

Single agent \acrshort{rl} applies when we want to find the best counter strategy of player $\Opt\in\PlayerSet$ in a \acrshort{mpg}. Model-based \acrshort{rl} applies when $\Opt$'s strategy is deterministic\footnote{For the deterministic case, \acrshort{rl} is an overkill. We can fallback to negative cycle finding.} or fractional. If $\Opt$'s strategy is complicated or unknown, we can only use Model-free \acrshort{rl} methods.

While our ultimate objective is a model that plays good enough irrespective of the opponent's strategy, our implementation does support for \acrshort{rl}-based strategy countering, albeit it does need retraining for every \acrshort{mpg} instance.

\subsubsection{Multi Agent}
When we have multiple agents with potentially conflicting objectives, we call this multi-agent \acrshort{rl}.

As the single agent case, each agent in the \acrshort{sg} may or may not know the the underlying model\footnote{Model as in the \acrshort{mdp} sense.}.

In our case, a \acrshort{mpg} can be modeled in the \acrshort{rl} setting as a turn-based, two-player, zero-sum \acrfull{sg}\footnote{Also known as Markov Game} \cite{StochasticGames}.

We will use the \acrshort{sg} formalism for the self-playing part.
\subsection{\acrshort{rl} formalisation of a \acrshort{mpg}}
To formalise a \acrshort{mpg} in \acrshort{rl} setting, we have two formalisms. Before diving into that, we will use the \acrshort{mpg} definition in section \ref{section:Formalisation:MPG}:
\begin{equation*}
	G=(\VertexSet,\EdgeSet,W,s,p)
\end{equation*}

\subsubsection{Instance based formalism}
In this formalism, we the \acrshort{sg} is directly defined by the \acrshort{mpg}.

That is, th

\subsubsection{Global formalism}
In this formalism, we will augment the \acrshort{sg} to account to the space of all \acrshortpl{mpg}.

This will require the following definition:
\begin{definition}
	The set of all \acrshortpl{mpg}, denoted by $\mathbb{M}$ is the set generated by $G=(\VertexSet,\EdgeSet,W,\PlayerSet,s,p)$ where:
	\begin{enumerate}
		\item $\varnothing\subsetneq V \subset_{\text{finite}} \mathbb{N}$
		\item $E \subseteq V\times V$ with the sinkless requirement:
		\begin{equation*}
			\forall u \in V,\quad \Adj u \neq \varnothing
		\end{equation*}
		\item $W\in\mathscr{F}(E,[-1,1])$
		\item $s\in \VertexSet.$
		\item $p\in \PlayerSet$
	\end{enumerate}
\end{definition}
	This set contains all \acrshortpl{mpg} up to isomorphism and positive rescaling.
	
	In this sense, we can play $G$ in $\mathbb{M}$ as follow:
	\begin{itemize}
		\item We set $G\in \mathbb{M}$ as the starting state.
		\item Let $G=(\VertexSet,\EdgeSet,W,\PlayerSet,s,\Opt)$ be the current node, then $\Opt\in\PlayerSet$ is the current player, his available actions are $\Adj s$
		\item Suppose that $\Opt$ chose action $t\in \Adj s,$ then he gets a payoff of $W(s,t),$ and the state transitions to $G'=(\VertexSet,\EdgeSet,W,\PlayerSet,t,\bar{\Opt})$
	\end{itemize}
	In fact, the game that we have just proposed is the same as the following \acrshort{mpg} $\mathscr{G}=(\mathbb{M},\mathbb{E},\mathbb{W},\VertexSet,G,p)$ with:
	\begin{align*}
		\mathbb{E}&=\left\{(\VertexSet,\EdgeSet,W,\PlayerSet,s,\Opt)\rightarrow (\VertexSet,\EdgeSet,W,\PlayerSet,t,\bar{\Opt})/\quad (s,t) \in E\right\} \\
		\forall G\rightarrow G',\quad \mathbb{W}(G,G')&=W(s,t) \quad \text{Where} \ \begin{cases}
			s & \text{is the starting vertex of} \ G \\
			t & \text{is the starting vertex of} \ G'
		\end{cases} 
	\end{align*} 
	Now, it is trivial to see that the \acrshort{mpg} $\mathscr{G}$ is equivalent to $G,$ and here we put all this theory into effect: \textbf{The \acrshort{rl} methods will be done in $\mathscr{G}$ instead of $G$.}
	
	\subsection{Benefits of global formalism}
	At first glance, it seems that we have only complicated the matter. In fact, this is justified as we can only effectively apply function approximation methods \acrshort{rl} using the global formalism.
	
	To clarify our point, we recall that in chapter \ref{section:ModelDesign}, we designed a model $\mathcal{M}_\Theta$ that takes a \acrshort{mpg} $G$ and outputs a strategy $\Pi_\Theta$ and an evaluation $v_\Theta.$ We will use \acrshort{rl} to refine $\Theta$ so that: 
	\begin{itemize}
		\item $v_\Theta(G)$ approaches the value of the game\footnote{The definition is stated in section} $v(G).$
		\item $\Pi_\Theta(G)$ approaches an optimal strategy $\Pi$ of game $G.$
	\end{itemize}
	Now, with the right choice of \acrshort{rl} algorithm, we expect that for any close enough\footnote{To formalise this idea, we need to define a metric space in $\mathbb{G}.$ This is beyond the scope of this report, but the point still holds.} game $G':$
	\begin{itemize}
		\item $v_\Theta(G')$ will also approach the value of the game $v(G').$
		\item $\Pi_\Theta(G')$ will also approach an optimal strategy $\Pi$ of game $G'.$
	\end{itemize}
	This is crucial: \textbf{We are training $\mathcal{M}_{\Theta}$ using \acrshort{rl} so that it generalises to any \acrshort{mpg}.}
	
	In contrast, if we just use the instance based formalism, the underlying model can only take the current position as information, and thus it does only generalises to solving the same game, but with only varying starting positions.
	\subsection{Generating random \acrshortpl{mpg}}
	This is the last thing to be addressed before implementing the self-playing pipeline.
	
	To expect our model to generalises well to any \acrshort{mpg}. We should generate the games using a suitable distribution. To do that, we introduced two additional distributions:
	\subsubsection{$\mathcal{D}^\bullet_{p,\text{US}}$ distribution}
	\begin{itemize}
		\item Let $\mathcal{N}$ be a non-empty finite subset of $\mathbb{N}.$ 
		\item Let $\mathcal{P}$ be a mesurable subset of $[0,1].$
		\item For a mesurable set $S,$ we will denote by $\mathcal{U}(S)$ the uniform distribution on $S.$
	\end{itemize}
	The $\mathcal{D}^\bullet_{p,\text{US}}(\mathcal{N},\mathcal{P})$ is a mixture distribution defined as follow:
	\begin{equation*}
		\mathcal{D}^\bullet_{p,\text{US}}(\mathcal{N},\mathcal{P}) = \mathcal{D}^\bullet(N,P) \quad \text{where}\ \begin{cases}
			N &\sim \mathcal{U}(\mathcal{N}) \\
			P &\sim \mathcal{U}(\mathcal{P})
		\end{cases}
	\end{equation*}
	To sample an element from $\mathcal{D}^\bullet_{p,\text{US}}(\mathcal{N},\mathcal{P})$:
	\begin{enumerate}
		\item Select the graph size $N\in\mathcal{N}$ uniformly at random.
		\item Select the edge probability $p\in\mathcal{P}$ uniformly at random.
		\item Build $G\sim \mathcal{D}^\bullet(N,P)$ using algorithm \ref{alg:Dnp_Fast} and degree rejection method defined at section \ref{section:Dataset:Sinkless:DegreeRejection}.
			
	\end{enumerate} 
	
	\subsubsection{$\mathcal{D}^\bullet_{c,\text{US}}$ distribution}
	\begin{itemize}
		\item Let $\mathcal{N}$ be a non-empty finite subset of $\mathbb{N}.$ 
		\item Let $\mathcal{C}$ be a non-empty finite subset of $\mathbb{N}.$
	\end{itemize}
	The $\mathcal{D}^\bullet_{c,\text{US}}(\mathcal{N},\mathcal{C})$ is a mixture distribution defined as follow:
	\begin{equation*}
		\mathcal{D}^\bullet_{c,\text{US}}(\mathcal{N},\mathcal{C}) = \mathcal{D}^\bullet(N,\tfrac{C}{N}) \quad \text{where}\ \begin{cases}
			N &\sim \mathcal{U}(\mathcal{N}) \\
			C &\sim \mathcal{U}(\mathcal{C})
		\end{cases}
	\end{equation*}
		To sample an element from $\mathcal{D}^\bullet_{c,\text{US}}(\mathcal{N},\mathcal{C})$:
	\begin{enumerate}
		\item Select the graph size $N\in\mathcal{N}$ uniformly at random.
		\item Select the expected degree $c\in\mathcal{C}$ uniformly at random.
		\item Build $G\sim \mathcal{D}^\bullet(N,\tfrac{C}{N})$ using algorithm \ref{alg:Dnp_Fast} and degree rejection method defined at section \ref{section:Dataset:Sinkless:DegreeRejection}.
	\end{enumerate} 
\subsubsection{Importance}
In chapter \ref{section:Dataset}, we designed and implemented a graph generation pipeline, then a graph annotation pipeline for dense graphs and later sparse graphs.

Both datasets $\mathcal{D}^\text{dense}$ and $\mathcal{D}^\text{sparse}$ contain the evaluation of each game and its respective optimal strategy. We can use both datasets for evaluation purposes, and for that we will make sure that our learns from the same distribution.

Following our implementation in chapter \ref{section:Dataset}: 
\begin{itemize}
	\item Each generated dense graph follow $\mathcal{D}^\bullet_{p,\text{US}}(\mathcal{N},\mathcal{P}).$ with $\mathcal{N}$ and $\mathcal{P}$ defined on section \ref{section:MPG:Generation:Distribution}, at the \textbf{Dense Graphs} paragraph.
	\item Each generated sparse graph follow $\mathcal{D}^\bullet_{c,\text{US}}(\mathcal{N},\mathcal{C})$ with $\mathcal{N}$ and $\mathcal{C}$ defined on section \ref{section:MPG:Generation:Distribution}, at the \textbf{Sparse Graphs} paragraph.
\end{itemize} 
	
\subsection{Considered \acrshort{rl} algorithm}
Now, all the technicalities about the formalisation of \acrshort{mpg} in \acrshort{rl} setting are resolved. Also, we have a complete model $\mathcal{M}$ as a result of chapter \ref{section:ModelDesign}.

The remaining part is the choice of the \acrshort{rl} itself. As hinted by previous sections, we will implement a self-playing system based on Alpha Zero \cite{AlphaZero}.

This system will indefinitevely:
\begin{itemize}
	\item Generate random instances of a \acrshort{mpg}, and annotate them with the model $\mathcal{M}_{\Theta}$ using self-play.
	\item Sample a dataset $\mathcal{D}$ from the generated annotated games, and fit the model $\mathcal{M}_{\Theta}$ to get a refined model $\mathcal{M}_{\Theta'}.$
	\item Evaluate the model against a fixed set of opponents.
\end{itemize}
With all that said, only the self-playing system is remaining. Thus in the upcoming sections, we will:
\begin{enumerate}
	\item Formalise the self-playing approach.
	\item Fully implement it.
	\item Deploy it in a \acrshort{hpc} cluster.
	\item Execute the whole pipeline and fine-tune the model.
\end{enumerate}
\newpage
\section{Monte Carlo Tree Search}
\acrfull{mcts} is an advanced algorithm in decision making that appeared in \citedate{MCTSOriginal}. \citeauthor{MCTSOriginal} first coined the term \acrshort{mcts} as application of Monte-Carlo methods to game-tree search \cite{MCTSOriginal}. It is an important algorithm for adversial search. that had a huge success at improving the competence of Go engines to a master level \cite{GoMaster}. This alone was a feat considering that previous state of the are alpha-beta implementations were unable to get past beginner level. 

Much development occured to \acrshort{mcts}, which made it a little hard to define. \citeauthor{RLIntroduction} gave the following definition \cite[section.~8.8]{RLIntroduction}: \textit{``\acrfull{mcts} is a planning algorithm that accumulates value estimates obtained from Monte Carlo simulations in order to successively direct simulations towards more highly-rewarded trajectories."}

This definition alone is unsatisfactory, \citeauthor{AIModernApproach} gave an intuitive explanation \cite[Section~6.4]{AIModernApproach} of how \acrfull{mcts} works for a game. And we will base this section on his work. 
 
Essentially, \acrshort{mcts} consists of the following $4$ steps:
\subsection{Selection}
Starting at the root, we choose a child following a \textbf{selection policy} $\Pi^{\text{selection}}$, and repeat the step until arriving to a leaf node.
\begin{algorithm}
	\caption{Selection algorithm for \acrshort{mcts}}\label{alg:MCTSSelection}
	\begin{algorithmic}
		\Require $r$ the root of the \acrshort{mcts}.
		\Require $\Pi^{\text{selection}}$ the current selection policy..
		\Ensure $s$ a leaf of the \acrshort{mcts}.
		\State $s\leftarrow r$
		\While{$\Not \ \text{leaf}(r)$}
			\State  $s\leftarrow \Pi^{\text{selection}}(r)$
		\EndWhile
		\State \Return $s$
	\end{algorithmic}
\end{algorithm}

\subsection{Expansion}
At the selected leaf $s$, we grow the \acrshort{mcts} by adding a new child $t$ that is a result of some action $a$ from $s.$

\begin{algorithm}
	\caption{Expansion algorithm for \acrshort{mcts}}\label{alg:MCTSExpansion}
	\begin{algorithmic}
		\Require $s$ a leaf \acrshort{mcts}.
		\Require $H$ the mean reward of each node in the \acrshort{mcts}
		\Require $T$ the number of simulations for each node in the \acrshort{mcts} 
		\Ensure $t$ a new child of $s$
		\State  $C\leftarrow \text{expand\_children}(s)$ \Comment{Generate a set of children from $s$}
		\For {$c \in C$}
			\State $T(c)\leftarrow 0$ \Comment{$c$ is currently visited $0$ times}
			\State $H(c)\leftarrow \boldsymbol{0}$ \Comment{The reward of each player at $c$ is initialized to $0$}
		\EndFor
		\State \text{$t\leftarrow \text{choose}(C)$} \Comment{Choose a child from $C$}
		\State \Return $t$
	\end{algorithmic}
\end{algorithm}

\subsection{Simulation}
Starting from the new child $t,$ we perform a complete playout, that is, we choose moves for all players according to the playout policy $\Pi^{\text{playout}}.$
\newline The generated sequence of nodes in the playout will not be recorded in the \acrshort{mcts}
\begin{algorithm}
	\caption{Simulation algorithm for \acrshort{mcts}}\label{alg:MCTSSimulation}
	\begin{algorithmic}
		\Require $t$ the new child of the \acrshort{mcts}.
		\Require $\Pi^{\text{playout}}$ the current playout policy..
		\Ensure $W$ the cumulative rewards after the playout for each player.
		\State  $s\leftarrow \text{state}(t)$ \Comment{get the current state}
		\State  $p\leftarrow \text{state}(t)$ \Comment{get the current player}
		\State $W\leftarrow \boldsymbol{0}$ 
		\While{$\Not \ \text{terminal}(s)$}
		\State  $a\leftarrow \Pi^{\text{playout}}(s,p)$
		\State $(s,p,U) \leftarrow \text{apply}(s,a)$ \Comment{Apply action $a$ at state $s$, and getting reward $U$}
		\State $W\leftarrow W+U$
		\EndWhile
		\State \Return $W$
	\end{algorithmic}
\end{algorithm}


\subsection{Backpropagation}
Once a simulation is complete, we use its result to update the all search nodes going from $t$ up to the root.
\begin{algorithm}[h]
	\caption{Backpropagation algorithm for \acrshort{mcts}}\label{alg:MCTSBackpropagation}
	\begin{algorithmic}
		\Require $t$ the new child of the \acrshort{mcts}.
		\Require $W$ the reward after the simulation.
		\Require $W'$ the cumulative reward up to node $t$
		\Require $H$ the mean reward of each node in the \acrshort{mcts}
		\Require $T$ the number of simulations for each node in the \acrshort{mcts}
		\While{$\Not \ \text{root}(s)$}
		\State $T(s)\leftarrow T(s)+1$ \Comment{Increment the number of visits of $s$}
		\State $H(s)\leftarrow H(s) - \frac{W+W'-H(s)}{T(s)}$  \Comment{Update the mean reward at $s$}
		\State $s\leftarrow \text{parent}(s)$ \Comment{Go to the parent node}
		\EndWhile
		\State \Return $W$
	\end{algorithmic}
\end{algorithm}
\FloatBarrier
\subsection{Wrapping up}
Once we have the $4$ elements of \acrshort{mcts}, the full algorithm will be just repeating these steps in order until certain desired condition is satisfied. In practice, most implementations use the following conditions for halting the  algorithm:
\begin{itemize}
	\item Execution time exceeds some threshold $T$
	\item Total number of simulations (at root) exceeds some threshold $N$
\end{itemize}

\begin{algorithm}
	\caption{\acrshort{mcts} Algorithm}\label{alg:MCTS}
	\begin{algorithmic}
		\Require $\text{state}$ some state of the game
		\Ensure An action $a\in \text{actions}(\text{state})$
		\State $r\leftarrow \text{tree}(\text{state})$
		\While {Exit condition not satisfied}
		\State $s\leftarrow \text{select}(r)$
		\State $t\leftarrow \text{expand}(s)$
		\State $W\leftarrow \text{simulate}(t)$
		\State $\text{backpropagation}(t,W)$
		\EndWhile
		\State \Return $a\leftarrow \displaystyle \argmax_{a\in \text{actions}(\text{state})}T(a)$ \Comment{Get the most visited action}
	\end{algorithmic}
\end{algorithm}
\begin{figure}
	\centering
	\includegraphics[width=0.75\textwidth]{Figures/MCTS.png}
	\caption{One iteration of a \acrshort{mcts} algorithm}
\end{figure}

\section{Model based \acrshort{mcts}}
\label{section:RL:ModelBasedMCTS}
While we explained how \acrshort{mcts} works in the previous section, some components were left without much discussion:
\begin{enumerate}
	\item The selection policy $\Pi^{\text{selection}}$
	\item The playout policy $\Pi^{\text{playout}}$
	\item Children creation
\end{enumerate}
This was deliberate, as the full power of $\acrshort{mcts}$ requires a good choice of the components, especially $\Pi^{\text{selection}}$ and $\Pi^{\text{playout}}.$

In our case, both functions will be based on the model $\mathcal{M}$ that we designed in chapter \ref{section:ModelDesign} and illustrated in figure \ref{fig:ModelArchitecture}.

Remember that our model $\mathcal{M}$ takes a \acrshort{mpg}\footnote{In fact, it takes a batch of \acrshortpl{mpg}, but this can be easily migitated.}, and produces two outputs:
\begin{enumerate}
	\item The predicted evaluation $v.$
	\item The predicted strategy $\Pi.$
\end{enumerate}
Both of these terms will be used in the \acrshort{mcts}.

\subsection{Create children}
In this version of \acrshort{mcts}, create children will return the list of all adjacent states. It does also apply an optional Dirichlet noise $\mathcal{D}(\alpha)$ if the current node is root.

\begin{algorithm}[H]
	\caption{Create children}\label{alg:CreateChildren}
	\begin{algorithmic}
		\Require $h$ current
		 node
		 \Require Dirichlet noise parameter $\alpha\in\mathbb{R}_+$
		 \Require Exploration noise parameter $\varepsilon\in[0,1]$
		 \Require Prior distribution $\Pi$
		\Ensure A list of children $C$
		\Ensure Optionally update $\Pi$
		\State $G=(V,E,W,s,p)$ is the underlying \acrshort{mpg} related to node $h$
		\State $m\leftarrow \lvert \Adj s \rvert$
		\If {$\text{root}(h)$}
			\State generate $\nu\sim \mathcal{D}(\alpha \ones_m)$ \Comment{$\ones_m \in \mathbb{R}^m$ is the vector of ones}
			\For {$(v,x)\in \zip(\Adj s,\nu)$}
				\State $\Pi(v)\leftarrow (1-\varepsilon)\Pi(v) + \varepsilon x$
			\EndFor
		\EndIf
		\State $C\leftarrow \varnothing$
		\For {$t\in \Adj s$}
			\State $G'\leftarrow (V,E,W,t,\bar{p})$
			\State $\text{append}(C,G')$
		\EndFor
		\State \Return $C$
	\end{algorithmic}
\end{algorithm}

\subsection{Playout Policy}
\label{section:RL:ModelBased:PlayoutPolicy}
In Alpha Zero's implementation of \acrshort{mcts} \cite{AlphaGo}, the playout policy is exactly the predicted distribution $\Pi.$

Also, the simulation algorithm \ref{alg:MCTSSimulation} was modified to only play one move:

\begin{algorithm}[H]
	\caption{Alpha Zero \acrshort{mcts} Simulation}\label{alg:AZSimulation}
	\begin{algorithmic}
		\Require $t$ the new child of the \acrshort{mcts}.
		\Require $\mathcal{M}$ the neural network model
		\Ensure $W$ the cumulative rewards after the playout for each player.
		\State  $G\leftarrow \text{MPG}(t)$ \Comment{get the current state}
		\State  $p\leftarrow \text{player}(t)$ \Comment{get the current player}
		\If {$p$ is $\Min$}
			\State $G\leftarrow \bar{G}$
		\EndIf
		\State $(v,\Pi)\leftarrow \mathcal{M}(G)$
		\If {$p$ is $\Min$}
			\State $v\leftarrow -v$
		\EndIf
		\State $W\leftarrow v$
		\Return $W$
	\end{algorithmic}
\end{algorithm}
\FloatBarrier
\subsection{Selection Policy}
\label{section:RL:ModelBased:SelectionPolicy}
\begin{algorithm}[H]
	\caption{model-based \acrshort{mcts} playout policy}\label{alg:SelectionPolicy}
	\begin{algorithmic}
		\Require $h$ current node 
		\Require $C\in\mathbb{R}_+$ exploration parameter
		\Ensure An child \acrshort{mpg} $G'$
		\State $Z\leftarrow \boldsymbol{0}$
		\For {$c\in \text{children}(h)$}
			\State $Z(c)\leftarrow \PUCT(c,h,\Pi(c),C)$
		\EndFor
		\State \Return $\displaystyle c \leftarrow \argmax_{c\in \text{children}(h)}Z(c)$
	\end{algorithmic}
\end{algorithm}
Here $\PUCT$ is defined as follow:
\begin{equation}
	\label{eqn:PUCT}
	\PUCT(c,h,\Pi(c),C) = H(c) + C\cdot \Pi(c) \cdot \frac{\sqrt{N(h)}}{N(c)+1}
\end{equation}
Here $H(c)$ is the reward of the current player at node $c$.

The equation \eqref{eqn:PUCT} was defined in the Alpha Go paper \cite{AlphaGo} as a model-based variant of $\UCT$ \cite{UCT}. We show the latter's expression to show the differnece between vanilla \acrshort{mcts} and model-based \acrshort{mcts}:
\begin{equation*}
	\UCT(c,h,C) =  H(c) + C\cdot \frac{\log {N(h)}}{N(c)+1}
\end{equation*}

In fact, the additional term $\Pi(c)$ is used to weight the children selection by the prior distribution $\Pi,$ and we believe that using the square root instead of the logarithmic function was deliberate to boost exploration.

\subsection{Wrapping up}
Our model-based \acrshort{mcts} is heavily inspired from the one used at Alpha Zero.

It uses the algorithm \ref{alg:MCTS} with a threshold number of iterations $N$ as exit condition\footnote{This applies to the model training phase. While deploying the algorithm, we can use a different exit condition.}. It avoids complete rollouts, and instead uses the simulation algorithm \ref{alg:AZSimulation} that uses the model $\mathcal{M}$.

Also, the predictions of model $\mathcal{M}$ are directly used in the selection policy defined at section \ref{section:RL:ModelBased:SelectionPolicy} and the playout policy defined at section \ref{section:RL:ModelBased:PlayoutPolicy}.
\subsection{Updating model}
Once the \acrshort{mcts} algorithm terminates. We get for each node a new estimates of both the strategy and the evaluation.
\subsubsection{Evaluation estimate}
The term $H$ used in the \acrshort{mcts} is used for the expected reward for the algorithm. Now as each node $t$ of the tree encodes a \acrshort{mpg} $G,$ $H(t)$ will be the updated estimate of the evaluation:
\begin{equation*}
	v(G)=\mathcal{M}(G)_{\text{evaluation}}
\end{equation*}

For that, we will denote the updated value as $v'(G)=H(t)$

\subsubsection{Strategy estimate}
To illustrate our point, we will need the following definitions:
\begin{itemize}
	\item Let $t$ be a node.
	\item Let $G=(V,E,W,s,p)$ be the \acrshort{mpg} encoded by $t.$
	\item Let $a\in \Adj s$ be an action eligible from $t.$
	\item Let $c$ be the\footnote{It is unique as an \acrshort{mpg} is deterministic. For a stochastic \acrshort{mpg}, which is out of the scope of this report, one should do a more careful analysis.} node resulting from applying $a$
\end{itemize}

Now, we want an esimtate of the strategy implied by the \acrshort{mcts}. In fact, the  term $N$ used for total explorations helps us to achieve this as following:
\begin{equation}
	\Pi'(G)(a) = \frac{N(c)}{N(t)} 
\end{equation}
Here, $\Pi'(G)$ will be the refined estimate of $\Pi(G)$

\subsubsection{Refitting}
\begin{itemize}
	\item Let $\Theta$ be the learnable parameters of $\mathcal{M}_\Theta=(v_{\Theta},\Pi_{\Theta})$
	\item Let $G^{(1)},\dots,G^{(m)}$ a batch of \acrshortpl{mpg}.
\end{itemize}
We will execute the \acrshort{mcts} on each game $G^{(i)}$ with the model $\mathcal{M}_{\Theta}$ to get a finer estimate of the evaluation $v'_{\Theta}(G^{(i)})$ and the strategy $\Pi'_{\Theta}(G^{(i)}).$

Once, we generate enough samples with \acrshort{mcts}. We train the model $\mathcal{M}_{\Theta}$ against $\mathscr{D}$ defined as follow:
$$
\mathscr{D}=\left[\left(G^{(1)},v'_{\Theta}(G^{(1)}),\Pi'_{\Theta}(G^{(1)})\right), \dots,\left(G^{(m)},v'_{\Theta}(G^{(m)}),\Pi'_{\Theta}(G^{(m)})\right)\right]
$$ 
The training uses the objective function defined on.

This will give us new learnable parameters $\Theta'$, and thus a new model $\mathcal{M}_{\Theta'}.$ With that, we will repeat the process indefinitively.


\section{Services}
\subsection{Rationale}
In section \ref{section:RL:ModelBasedMCTS}, we discussed our model-based implementation of \acrshort{mcts}. Roughly speaking, it can be divided into two main components that are repeated indefinitevely:
\begin{enumerate}
	\item Executing \acrshort{mcts} using model $\mathcal{M}_{\Theta^{(i)}}$ to generate a dataset $\mathcal{D}$: $$
	\mathscr{D}=\left[\left(G^{(1)},v'_{\Theta}(G^{(1)}),\Pi'_{\Theta}(G^{(1)})\right), \dots,\left(G^{(m)},v'_{\Theta}(G^{(m)}),\Pi'_{\Theta}(G^{(m)})\right)\right]
	$$
	\item Fit $\mathcal{M}_{\Theta^{(i)}}$ against $\mathscr{D}$ to get a better model $\mathcal{M}_{\Theta^{(i+1)}}$
\begin{figure}[ht]
	\centering
	\includegraphics[width=0.85\textwidth]{Figures/AlphaZero.jpg}
	\caption{Main steps of Alpha Zero}
\end{figure}
\end{enumerate}

Each one of these steps is computationnally intensive. And for that, we will use the \text{seperation of concerns} principle to split them into distinct services.\footnote{This seperation, while not necessary, is benificial as it gives the freedom to scale the self-play pipeline.}
\subsection{Learner}
The learner, as its name suggests, is the service that trains the model $\mathcal{M}$ upon receiving enough samples from the actors\footnote{Which we will define in the next section.}

\subsection{Actor}
An actor, is a service designed for generating samples to the learner by playing agains itself using model-based \acrshort{mcts}

In contrast to the learner, which is unique by design. The self-play pipeline can have many actors, and this is required in our case as it accelerate the execution of the whole pipeline.

\subsection{Evaluator}
This service, is a third component that is used to evaluate the performance of model-based \acrshort{mcts} against a fixed player.

Like the actor, a self-play pipeline can have many evaluators, and this is also required in our case as it give performance metrics against a wide range of players.

Now, we defined the services and talked briefly about their relations. The figure below will describe the communication between the services:
\begin{figure}[H]
	\centering
	\includegraphics[width=0.7\textwidth]{Figures/ServiceRelations.png}
	\caption{Relation between different services\label{fig:RelationServices}}
\end{figure}
\FloatBarrier


\section{Self-play pipeline}
\subsection{Decoupling service communication}
In the previous section, we seperated logically between the different components of the self-play system.

Now, this seperation will give us the benefit of having multiple actors and evaluators per instance. While this is beneficial, it is only vertically scalable.

In fact, the interactions in figure \ref{fig:RelationServices} hints that the relation between components is tightly coupled. This is not the case, as we applied the \textbf{broker} design pattern to further decouple the communication between the different components of the system.

To do that, we introduced:
\begin{itemize}
	\item The \texttt{Orchestrator}: Which accumulate the generated annotated games and send them to the learner. It is a part of the actor service.
	\item The \texttt{ReplayBuffer}: Which loads a dataset from the annotated games that were generated by actors. It is part of the learner service.
	\item The \texttt{ModelBroadcaster}: It broadcast the new model to all actors and evaluators. It is part of the learner service.
\end{itemize}
The figure below details the relation between all main components of the pipeline:

\begin{figure}[H]
	\centering
	\includegraphics[width=0.85\textwidth]{Figures/ServiceDiagram.png}
	\caption{Class diagram of the services}
\end{figure}
\FloatBarrier
\subsection{Distributing the pipeline}
As we decoupled the communication, we only have to implement concrete definitions of the \texttt{Orchestrator}, \texttt{ReplayBuffer} and \texttt{ModelBroadcaster}.

Currently, we do have two set of implementations:
\begin{itemize}
	\item The first one defines the communication between the different services when they are centralized\footnote{Executed in the same machine.}.
	\item The second one defines the communication between the different services when they are distributed
\end{itemize}
The first implementation is present for compatibility reasons with the original implementation offered by \texttt{open\_spiel}. 

We distributed our pipeline by using the \acrshort{http} protocol\footnote{Note that this is not the only way to do that, the implementer is free to make his own implementation of \texttt{Orchestrator}, \texttt{ReplayBuffer} and \texttt{ModelBroadcaster}}. In fact, we used both \acrshort{rest} and \acrshort{grpc}.
Additionnally, both of \texttt{Orchestrator} and \texttt{ReplayBuffer} use \acrshort{grpc}, while \texttt{ModelBroadcaster} uses \acrshort{rest}. This will give us the opportunity to add horizontal scaling for the pipeline, by adding further actor nodes\footnote{Here node refers to a machine.} and evaluator nodes.

The next two sections are dedicated respectfully to the \acrshort{rest} server and the \acrshort{grpc} server.
\subsection{\acrshort{rest} servers}
Each service\footnote{Either actor or learner or evaluator} deploys a basic \textbf{FastAPI} \acrfull{rest} server used mainly for: 
\begin{itemize}
	\item Basic management: starting the pipeline or shutting the pipeline.
	\item Service discovery.
	\item Model broadcasting.
	\item Monitoring.
\end{itemize}
The learner actor serves as the master of the services. It will give instructions to the other services. It also serves as the gateway.

\begin{table}[h]
	\begin{tabularx}{\textwidth}{| p{3cm} | p{1.2cm} | p{1.7cm} | X |}
		\hline
		Route & Method & Service & Description \\
		\hline 
		\texttt{/start} &  GET & All & If the service is the learner, start the whole pipeline by requesting \texttt{/start} to all other services. Otherwise, start the requested service. \\
		\hline
		\texttt{/stop} &  GET & All & If the service is the learner, stop the whole pipeline by requesting \texttt{/stop} to all other services. Otherwise, stop the requested service. \\
		\hline
		\texttt{/heartbeat} & GET & All & Return a heartbeat. \\
		\hline
		\texttt{/replay\_buffer} & GET & Learner & Ouput information in JSON format about the \newline \textbf{Reverb} server. \\
		\hline
		\texttt{/config} & GET & Learner & Ouput the configuration file in JSON format. \\
		\hline
		\texttt{/discovery} & GET & Learner & Discover the services. \\
		\hline
		\texttt{/model} & POST & Actor, Evaluator & Sends the path of the newest version of the model. \\
		\hline
		\texttt{/stats} & GET & Actor, Evaluator & Manually get the statistics generated by the service.\\
		\hline
	\end{tabularx}
	\caption{Supported routes
		\label{table:ImplementedRoutes}}
\end{table}

\subsection{\acrshort{grpc} server}
The learner service also deploys a \textbf{Reverb}\footnote{Reverb is an efficient data storage and experience replay system for distributed reinforcement learning. It supports multiple data structure representations such as FIFO, LIFO, and priority queues. } server, which we will use for storing annotated games and sampling from them. It is based on \acrfull{grpc}\footnote{This is a recursive acronym.}, and designed primarly for efficiency. 

\begin{figure}[H]
	\centering
	\includegraphics[width=0.9\textwidth]{Figures/ReverbTable.png}
	\caption{Illustration of how Reverb works}
\end{figure}
\FloatBarrier
This server will act as an intermediate between the actors and the learner in the following sense:
\begin{itemize}
	\item The implemented \texttt{Orchestrator} will send batches of generated games to the server.
	\item The implemented \texttt{ReplayBuffer} will take samples from the \textbf{Reverb} server for training purposes.
\end{itemize}

\subsection{Tensorboard server}
\section{Implementation}

\section{Deployment}


\begin{figure}
	\centering
	\includegraphics[width=0.95\textwidth]{Figures/AlphaZero.png}
	\caption{Pipeline Architecture}
\end{figure}
\section{Configuration}
We externalized the configuration of our self-play pipeline, so that we could change the parameters and fine-tune the hyper-parameters without modifying the source code\footnote{Still, the pipeline needs to be restarted for changes to take effect.}. This gives us the possibility to play experiments until getting the desired computational and prediction performances.

We created a \acrshort{yaml} configuration file with the name \texttt{configuration.yml} that we already introduced in section \ref{section:ModelDesign:Configuration}. It will be used by all services\footnote{In a future iteration, we plan to split this file per service.}. When we completed the externalization process, we found out that we need also to set the file as a template so we can change it dynamically by environment variables. This is beneficial as we do have many instances of the same services that only differ by some minor hyper-parameters.

To achieve this, we introduced \textbf{Jinja} templating into the file, and set the name of the template configuration file as \texttt{configuration.yml.j2}. The templated configuration variables includes:
\begin{itemize}
	\item \texttt{path}: The directory of the pipeline
	\item The instance name of the service.
	\item The port used by the services \acrshort{rest} server.
	\item Type of the opponent for an evaluator service.
	\item The address and port of the \acrshort{grpc} service.
\end{itemize}

In the remaining of this section, we will describe the remaining configurations except the training and model configurations as they were already discussed in figures \ref{fig:ModelConfiguration} and \ref{fig:TrainingConfiguration}.

\subsection{Replay buffer}
Our pipeline supports two kinds of replay buffer. A ``\texttt{local}" one, that we can use in a centralized system. And a ``\texttt{grpc}" one\footnote{Which is the \textbf{Reverb} server.}, which is the one we are using.

The figure below shows the configuration parameters of the replay buffer:
\begin{figure}[ht]
	\centering
	\includegraphics[width=0.95\textwidth]{Figures/ReplayBufferConfiguration.png}
	\caption{Replay buffer section of the configuration file \label{fig:ReplayBufferConfiguration.png}}
	\small{Note: while this is a recent version of the configuration file, we may tweak it further for fine-tuning purposes.}
\end{figure}
\FloatBarrier
%
\subsection{Services}
We also externalized the parameters of the services, so we can easily change their depoloyment options. This is shown by the figure below:
\begin{figure}[ht]
	\centering
	\includegraphics[width=0.95\textwidth]{Figures/ServicesConfiguration.png}
	\caption{Replay buffer section of the configuration file \label{fig:ServicesConfiguration.png}}
	\small{Note: while this is a recent version of the configuration file, we may tweak it further for fine-tuning purposes.}
\end{figure}
\FloatBarrier
\section{Execution}
\section*{Conclusion}
This chapter marks the culmination of all our research, proven results, implementations into a whole \acrshort{rl} system based on \acrshort{sp}.

First of all, we started by formalising our \acrshort{rl} approach, we formalised our \acrshort{mpg} as a \acrshort{sg} consistingly with the model we had built in chapter \ref{chapter:ModelDesign}. 

We also defined two additional \acrshort{mpg} distributions that will be used on the \acrshort{sp} pipeline. The choice of both distributions is consistent with both datasets discussed in chapter \ref{chapter:Dataset}. This is intended so that we can evaluate our algorithms against optimal players.

Second of all, we formalised the \acrshort{mcts} approach, and followed Alpha Zero's approach to get a \acrshort{mcts} variant that uses the model defined in chapter \ref{chapter:ModelDesign}.

Later, we seperated functionalities of the \acrshort{sp} pipeline following Alpha Zero's approach. Then we distributed the whole system so it can be scaled both horizontally and vertically.

Once we had resolved all implementation issues, we deployed the \acrshort{sp} pipeline to a \acrshort{hpc} cluster, externalized the configuration of the whole system.

Finally, we executed the whole \acrshort{sp} pipeline as a proof of concept.
\chapter{Analyse}

For $x\in \mathcal{X},Y\subseteq\mathcal{X}^m,c\in I$, let $\text{MA}(x,Y,c)$ be defined as follow:
$$
\text{MA}(x,Y,c)\iff x\le \max Y+c
$$
\chapter*{Conclusion}
\addcontentsline{toc}{chapter}{Conclusion}

Durant ce stage, nous avons étudié les BNNs, implémenté une bibliothèque de BNNs basée sur tensorflow et larq qu'on a nommé binaryflow. \\
\\
Dans le premier chapitre, nous avons présenté \textbf{dB Sense} et ses activités. \\
\\
Dans le deuxième chapitre, nous avons présenté le problème de la grande complexité, introduit le concept des BNNs et proposé \textbf{binaryflow} comme notre solution pour les BNNs\\
\\
Dans le troisième chapiter, nous avons formalisé les BNNs, et nous avons décris leurs optimisations possible, et les problèmes rencontrés dans leur implémentation.\\
\\
Dans le quatrième chapitre, nous avons présenté des modèles BNNs, chacun en dérivant ses formules\\
\\
Dans le cinquième chapitre, nous avons donné notre implémentation de \textbf{binaryflow} en jusitifiant les paradigmes utilisés\\
\\
Dans le sixième chapitre, nous avons analysé 3 jeux de données en implémentant des modèles BNNs pour chacune de ces 3 jeux de données, et en comparant les performances de prédicition et la complexité temps et mémoire des modèles entraînés.\\
\\
Notre travail n'est qu'une petite introduction des BNNs. En effet, la liste des BNNs proposés dans la littérature est très vaste, et il existe plusieurs autres approches pour faciliter l'entraînement et l'interférence des BNNs que nous n'avons pas considéré vu les contraintes de stage, y parmi:
\begin{enumerate}
	\item Les méthodes d'optimisations discrètes
	\item Les optimiseurs dédiés au BNNs
	\item Les binarisations entraînables
	\item Les méthodes ensemblistes pour régulaliser les BNNs
\end{enumerate} 
De plus, nous avons réussi à vérifier l'optimisation de la multiplication matricielle (L'éxecution est parfois 30 fois plus rapide), mais nous n'avons pas pu intégrer cette optimisation aux modèles tensorflow. Et malgré que larq supporte lui même un déploiement optimisé, nous n'avons pas aussi pu l'exploiter puisque ce déploiement ne supporte que les processeurs ARMv8, et nous utilisons une machine avec un processeur d'architecture x86-64.\\
\\
Finalement, nous avons fait une petite intégration du code carbone comme une mésure de coût d'entraînement. Une fois le problème de déploiement est résoulu, nous recommenderions l'utilisation de ce même métrique pour estimer le coût d'interférence qui va justifier l'utilisation des BNNs.   
\appendix
\chapter{On Constraint Satistfaction Problems}
\label{appendix:CSP}
In the previous chapters, we described how the system works, without formalising the CSP approach.\newline
On this chapter, we will describe the CSP systems that we have used, with an equivalence proof between them.

\section{Constraint Satisfaction Problem}
\subsection{Definition}
A constraint satisfaction problem
\subsection{Assignment}
An assignment of a $\CSP(\mathcal{X},D,\Gamma)$ is a function $X:\mathcal{X}\rightarrow D$ such that, by replacing each $x\in\mathcal{X},$ by $X(x),$ all the constraints will evaluate to $\True$
\subsection{Polymorphism}
A function $F:\mathscr{F}(\mathcal{X},D)^k\rightarrow \mathscr{F}(\mathcal{X},D)$ is said to be a polymorphism if:
$$
\forall X_1,\dots,X_k \ \text{assignments of}\ \CSP(\mathcal{X},D,\Gamma),\quad F(X_1,\dots,X_k) \ \text{is also an assignment of}\ \CSP(\mathcal{X},D,\Gamma)
$$
Now, in the next section, we will define an important class of CSPs that is used to solve Mean Payoff Games, with the polymorphisms required for the solution algorithm's correctness

\section{Ternary Max Atom Systems}
\subsection{Definition}
\begin{itemize}
	\item Let $\mathcal{X}$ be a finite set of variables
	\item Let $D=I\cup \{-\infty\},$ with $I\subseteq \mathbb{R}$.
	\item  For $x,y,z\in \mathcal{X},c\in I$, let $\text{MA}_3(x,y,z,c)$ be defined as follow:
	$$
	\text{MA}_3(x,y,z,c)\iff x\le \max(y,z)+c
	$$
\end{itemize}
A ternary max atom system is $\CSP(D,\Gamma)$ where:
\begin{align*}
	\Gamma&=\left\{\text{MA}_3(x,y,z,c),\quad (x,y,z,c)\in \mathscr{R}\right\}\\
	\mathscr{R}&\subseteq \mathcal{X}^3\times I\\
	\mathscr{R}& \space \text{is finite}
\end{align*}
\subsection{Example}
An example of a ternary max atom system is the following $\CSP(D,\Gamma)$ with $D=\mathbb{Z}$ and $\Gamma$ represented as follow:
\begin{align*}
	x &\le \max(y,z)-1 \\
	y &\le \max(z,x)-1 \\
	z & \le \max(x,y)-1
\end{align*}


\section{Max Atom Systems}
\subsection{Definition}
\begin{itemize}
	\item Let $\mathcal{X}$ be a finite set of variables
	\item Let $D=I\cup \{-\infty\},$ with $I\subseteq \mathbb{R}$   
	\item For $x\in \mathcal{X},Y\subseteq\mathcal{X}^m,c\in I$, let $\text{MA}(x,Y,c)$ be defined as follow:
	$$
	\text{MA}(x,Y,c)\iff x\le \max Y+c
	$$
\end{itemize}

A max atom system is $\CSP(D,\Gamma)$ where:

\begin{align*}
	\Gamma&=\left\{\text{MA}(x,Y,c),\quad (x,Y,c)\in \mathscr{R}\right\}\\
	\mathscr{R}&\subseteq \mathcal{X}\times \left(\mathscr{P}(\mathcal{X}) \setminus \{\varnothing\}\right)\times I \\
	\mathscr{R}&\space \text{is finite}
\end{align*}

\subsection{$\text{MA} \le \text{MA}_3$}
\begin{itemize}
	\item Let $S=\CSP(\mathcal{X},D,\Gamma)$ a max atom system.
	\item Let $R\in \Gamma$
	\item Let $x\in \mathcal{X},Y\in\mathscr{P}(\mathcal{X}),c\in I$ such that $R=\text{MA}(x,Y,c)$ such that $\lvert Y \rvert >2$
\end{itemize}

\subsubsection{Recursive Reduction}
We will reduce the arity of $R$ as follow:
\begin{itemize}
	\item Let $y,z\in Y$ such that $y\ne z$
	\item We introduce a variable $w\notin \mathcal{X}$
	\item Let $\mathcal{X}'=\mathcal{X}\cup\{w\}$
	\item Let $Y'=(Y\cup \{w\})\setminus\{y,z\}$
	\item Let $R'=\text{MA}(x,Y',c)$
	\item Let $R_w=\text{MA}(w,\{y,z\},0)$
	\item Let $\Gamma'=(\Gamma\cup\{R',R_w\})\setminus \{R\}$
	\item Let $S'=\CSP(\mathcal{X}',D,\Gamma)$
\end{itemize}


We will prove that $S'$ is equivalent to $S.$

\paragraph{Implication}
Let $X:\mathcal{X}'\rightarrow D$ an assignment of $S'.$ It is trivial that by removing $X(w)$, $X_{\mid \mathcal{X}}$ is an assignment of $S$ 

\paragraph{Equivalence}
\begin{itemize}
	\item Let $X:\mathcal{X}'\rightarrow D$ such that $X_{\mid \mathcal{X}}$ is an assignment of $S.$
	\item We will set $X(w)=\max(X(y),X(z))$
\end{itemize}
Then, $X$ is an assignment of $S'$

\subsubsection{Induction}
Since the number of variables is finite, the arity of each constraint is finite. Also, as the the number of constraints is finite, Applying such reduction iteratively will eventually give a system $S^*$ equivalent to $S$ with:
\begin{itemize}
	\item $\mathcal{X}^*$ the set of variables with $\mathcal{X}\subseteq \mathcal{X}^*$ 
	\item $\Gamma^*$ is the set of constraints:
	\item Each constraint is of the form $\text{MA}(x,Y,c)$ with $x\in \mathcal{X}^*,Y\subseteq \mathcal{X}^*,c\in I$ with $\lvert Y\rvert \le 2$   
\end{itemize}
Now such system can be transformed to a ternary system $S_3$ as follow:
\begin{itemize}
	\item The set of variables is $\mathcal{X}^*$
	\item The domain is $D$
	\item For every relation $R=\text{MA}(x,Y,c)$ we map it to the relation $R_3=\text{MA}(x,y,z,c)$ as follow:
	\begin{itemize}
		\item If $\lvert Y \rvert=2$, then $y,z$ are the elements of $Y.$
		\item Otherwise, $\lvert Y \rvert=1,$ and $y=z$ are the same element of the singleton\footnote{A set with only one element} $Y.$
	\end{itemize}
	
\end{itemize}


It is trivial that $S^*$ is equivalent to $S_3.$
With that, $S$ is equivalent to $S_3.$


\begin{algorithm}
	\caption{Converting a Max Atom System to Ternary Max Atom System}\label{alg:MaxAtomToTernaryMaxAtom}
	\begin{algorithmic}
		\Require $S$ an $N$-ary Max Atom system
		\Ensure $S'$ a ternary Max Atom system  
		\State $S'\leftarrow \varnothing$
		\State $H\leftarrow\varnothing$ \Comment{$H$ is a symmetric map between variable,variable to variables}
		\State $V\leftarrow \text{Variables}(S)$  \Comment{$V$ is a set containing all variables}
		\For {$\mathcal{C}\in S$}
		\Comment{Iterate over constraints}
		\State $c$ is the constant in the right hand side of $\mathcal{C}$
		\State $Y$ is the variables in the right hand side of $\mathcal{C}$
		\State $x$ is the variable in the left hand side of $\mathcal{C}$
			\While{$\lvert Y \rvert > 2$}
				\State $y\leftarrow \pop(Y)$
				\State $z\leftarrow \pop(Y)$
				\If {$(y,z) \notin \domain  H$}
					\State $w\leftarrow \text{newVariable}(V)$\Comment{Generate a new formal variable not included in $V$}
					\State $V\leftarrow V\cup\{w\}$
					\State $H(y,z)\leftarrow w$
					\State $H(z,y)\leftarrow w$
				\EndIf
				\State $w\leftarrow H(y,z)$
				\State $S'\leftarrow S'\cup\{\text{MA}(w,y,z,c)\}$
				\State $Y\leftarrow Y\cup\{w\}$
			\EndWhile
		\EndFor
		\State \Return $S'$
	\end{algorithmic}
\end{algorithm}
\subsection{Polymorphisms}
Two main family of polymorphisms are defined for Max Atom systems:
\begin{itemize}
	\item The max polymorphisms $M^{k}$ defined by:
	$$
	M^{k}(X_1,\dots,X_k)(x) = \max_{k\in\{1,\dots,k\}} X_k(x)
	$$
	\item The translation polymorphisms $T_{c}$ defined by:
	$$
	T_c(X)(x)=X(x)+c
	$$
\end{itemize}

\section{Min-Max System}

\begin{itemize}
	\item Let $\mathcal{X}$ be a finite set of variables
	\item Let $I$ be the domain of the variables.
	\item Let $D=I\cup \{-\infty\},$ with $I\subseteq \mathbb{R}$  
	\item For $x\in \mathcal{X},Y\subseteq\mathcal{X}^m,C\in I^m$, let $\text{MA}(x,Y,C)$ be defined as follow:
	$$
	\text{MA}(x,Y,c)\iff x\le \max (Y+C)
	$$
	\item For $x\in \mathcal{X},Y\subseteq\mathcal{X}^m,C\in I^m$, let $\text{MI}(x,Y,C)$ be defined as follow:
	$$
	\text{MI}(x,Y,C)\iff x\le \min (Y+C)
	$$
\end{itemize}

A min-max system is $\CSP(D,\Gamma)$ where:
\begin{align*}
	\Gamma&=\left\{O(x,Y,C),\quad (O,x,Y,C)\in \mathscr{R}\right\}\\
	\mathscr{R}&\subseteq \{\text{MA},\text{MI}\}\times\mathcal{X}\times \left(\mathcal{X}\times I\right)^+ \\
	\mathscr{R}&\  \text{is finite}
\end{align*}


\subsection{Transforming to Max Atom Systems}
A Max Atom system is trivially a Min Max system. So we will only prove the latter implication.

Let $S'=\CSP(D,\Gamma)$ be a Min Max system, and let:
\begin{itemize}
	\item $\Gamma_{\text{MI}}$ be the constraints that has $\text{MI}$ 
	\item $\Gamma_{\text{MA}}$ be the constraints that has $\text{MA}$
\end{itemize}
\subsubsection{Transforming $\text{MI}$ constraints}
For each $\text{MI}(x,Y,c)\in \Gamma_{\text{MI}}.$ we replace it with the following constraints:
$$
\Gamma^{x,Y,C}_{\text{MI}}=\left\{\text{MA}(x,\{y\},c),\quad y,c\in \zip(Y,C)\right\}
$$
\subsubsection{Tranforming $\text{MA}$ constraints}
For each $(y,c)\in \mathcal{X}\times I$ present in a max constraint of the system: \begin{itemize}
	\item We add a formal variable $z^{y,c}$ if $c\ne 0.$
	\item Else, we will simply represent by $z^{y,c}$ the variable $y.$
\end{itemize}
By denoting $Z^{Y,C}$ the following set:
$$
Z^{Y,C}=\{z^{y,c},\quad (y,c)\in \zip(Y,C)\}
$$
We build the following constraints:
$$
\Gamma^{x,Y,C}_{\text{MA}} = \left\{\text{MA}(x,Z^{Y,C},0)\right\}\cup\{\text{MA}(z^{y,c},\{y\},c),\quad (y,c)\in \zip(Y,C)\}
$$
\subsubsection{Building the Max Atom System}
Now, let:
$$
\Gamma'=\bigcup_{\text{OP}\in\{\text{MI},\text{MA}\}}\bigcup_{\text{OP}(x,Y,C)\in \Gamma_{\text{OP}}} \Gamma^{x,Y,C}_{\text{OP}}
$$
The system $\CSP(D,\Gamma')$ is an equivalent max system.
\subsubsection{Equivalence}
Let:
\begin{itemize}
	\item $\mathcal{X}' = \mathcal{X}\cup \mathcal{X}_{\text{Generated}}$ the augmented set of variables.
	\item $\mathcal{C}=\text{OP}(x,Y,C)\in \CSP(D,\Gamma)$ be a constraint.
	\item $X:\mathcal{X'}\rightarrow D$ an assignment of $\CSP(D,\Gamma').$
\end{itemize}
If $\text{OP}=\text{MI},$ it is trivial that $\text{OP}(x,Y,C)$ is equivalent to $\Gamma^{x,Y,C}_{\text{OP}}.$
\newline Otherwise, for each $(y,c)\in \zip(Y,C)$ we have:
$$
X(z^{y,c}) \le \max\{X(y)\}+c = X(y)+c
$$
Now, we also have:
$$
X(x) \le \max_{(y,c)\in\zip(Y,C)}X(z^{y,c}) +0 \le \max_{(y,c)\in \zip(Y,C)}\left(X(y)+c\right)
$$
With that, $X_{\mid \mathcal{X}}$ is an assignment of $\CSP(D,\Gamma)$
\begin{algorithm}
	\caption{Converting a Min-Max System to Max Atom}\label{alg:MinMaxToMaxAtom}
	\begin{algorithmic}
		\Require $S$ a Min-Max system
		\Ensure $S'$ an $N$-ary Max Atom system  
		\State $S'\leftarrow \varnothing$
		\State $H\leftarrow\varnothing$ \Comment{$H$ is a map between variable,offsets to variables}
		\State $V\leftarrow \text{Variables}(S)$  \Comment{$V$ is a set containing all variables}
		\For {$\mathcal{C}\in S$}
		 \Comment{Iterate over constraints}
		 \State $C$ is the constants in the right hand side of $\mathcal{C}$
		 \State $Y$ is the variables in the right hand side of $\mathcal{C}$
		 \State $x$ is the variable in the left hand side of $\mathcal{C}$
		\If {$\mathcal{C}$ is a min constraint}
			\State $S'\leftarrow S'\cup\left\{\text{MA}(x,\{y,y\},c),\quad (y,c)\in \zip(Y,C)\right\}$
		\Else
			\State $Z\leftarrow \varnothing$
			\For {$(y,c)\in \zip(Y,C)$}
				\If {$(y,c) \notin \domain{H}$}
					\State $z\leftarrow \text{newVariable}(V)$ \Comment{Generate a new formal variable not included in $V$}
					\State $V\leftarrow V\cup\{z\}$
					\State $H(y,c)\leftarrow z$
				\EndIf
				\State $S'\leftarrow S'\cup\{\text{MA}(H(y,c),\{y,y\},c)\}$
				\State $Z\leftarrow Z\cup \{H(y,c)\} $
			\EndFor
			\State $S' \leftarrow S' \cup \{\text{MA}(x,Z,0)\}$
		\EndIf
		\EndFor
	\end{algorithmic}
\end{algorithm}
\section{Solving Mean Payoff}
\subsection{Reduction to Min Max System}
In this section, we will solve the mean payoff by converting it to a equivalent min max system.
\newline The method we use is formalised in \cite{MPGMaxAtom}, and the equivalence is also proven in \cite{MPGMaxAtom}.
\begin{algorithm}
	\caption{Converting a Mean Payoff Game to a Min Max system}\label{alg:MPGToMinMax}
	\begin{algorithmic}
		\Require $G$ a Mean Payoff Game
		\Ensure $S$ an Min-Max system 
		\State $E\leftarrow E(G)$ \Comment{The edges of $G$}
		\State $V\leftarrow V(G)$\Comment{The variables of $G$}
		\State $W \leftarrow W(G)$ \Comment{The weight function of $G$}
		\State $P \leftarrow P(G)$ \Comment{The set of player of $G$}
		\For {$(u,p) \in V\times P$}
			\State $x\leftarrow (u,p)$
			\State $A\leftarrow \Adj(x)$
			\State $Y=\{(a,\bar{p}),\quad a\in A\}$
			\State $C\leftarrow W(A)$ \Comment{Calculating the weights element wise.}
			\If {$p$ is Max}:
				\State $\text{OP}\leftarrow \text{MA}$			
			\Else
				\State $\text{OP}\leftarrow \text{MI}$	
			\EndIf
			\State $S\leftarrow S\cup \left\{\text{OP}(x,Y,C)\right\}$
		\EndFor
	\end{algorithmic}
\end{algorithm}
\subsection{Arc Consistency}
\subsubsection{First Implementation}
At first, we took the original implementation of AC3 \cite[page.~171]{AIModernApproach} and modify it to support the max atom system:
\begin{algorithm}
	\caption{AC3 for Ternary Max Atom systems}\label{alg:AC3}
	\begin{algorithmic}
		\Require $\mathcal{C}$ a ternary Max Atom constraint
		\Require $\nu:V\rightarrow \mathscr{P}(D)$ the admissible values function
		\Require $Q$ a Queue of pending updates
		\Ensure $\nu$ update the admissible values function.
		\State	$x\leftarrow$ the left-hand side of $\mathcal{C}$
		\State	$(y,z)\leftarrow$ the right-hand side variables of $\mathcal{C}$
		\State	$c\leftarrow$ the right-hand side constant $\mathcal{C}$
		\State $Z\leftarrow [x,y,z]$
		\For {$o\in \{0,1,2\}$} \Comment{Iterate over all rotations}
			\State $\text{admissible}\leftarrow \False$
			\State $x\leftarrow Z[o]$
			\State $y\leftarrow Z[(o+1)\bmod 3]$
			\State $z\leftarrow Z[(o+2)\bmod 3]$
			\State $\mu\leftarrow \varnothing$ \Comment{Set of values to delete}
			\For {$a\in \nu(x)$}
				\For {$(b,c)\in \nu(y)\times \nu(z)$}
					\State $T\leftarrow[a,b,c]$
					\State $p\leftarrow T[(-o)\bmod 3]$
					\State $q\leftarrow T[(1-o)\bmod 3]$
					\State $q\leftarrow T[(2-o)\bmod 3]$
					\If {$p\le \max(q,r)+c$}
						\State $\text{admissible}\leftarrow \True$
					\EndIf
				\EndFor
				\If	{$\Not \text{admissible}$}
					\State $\mu\leftarrow \mu \cup \{a\}$
				\EndIf
			\EndFor
			\If {$\lvert \mu \rvert >0$}
				\State $\nu(x)\leftarrow \nu(x)\setminus \mu$
				\State $\text{append}(Q,\Adj x)$
			\EndIf

		\EndFor
	\end{algorithmic}
\end{algorithm}
\FloatBarrier
This version was very slow, took a considerable time even for Mean Payoff Games with less than $20$ vertices.
\subsubsection{Refinement}
We were able to make many simplifications to the algorithm by taking advantage of the symmetry\footnote{$\max$ is a symmetric function: $\max(x,y)=\max(y,x)$} of the system and the fact that it has translations and maximum as polymorphisms \cite{TropicalCSP}:
\begin{algorithm}
	\caption{AC3 Optimized for Ternary Max Atom systems}\label{alg:AC3Optimized}
	\begin{algorithmic}
		\Require $\mathcal{C}$ a ternary Max Atom constraint
		\Require $\nu:V\rightarrow D$ the maximum admissible value for each variable.
		\Require $Q$ a Queue of pending updates
		\Require $L=\inf D$ the smallest admissible value
		\Ensure $\nu$ update the maximum admissible values function.
		\State	$x\leftarrow$ the left-hand side of $\mathcal{C}$
		\State	$(y,z)\leftarrow$ the right-hand side variables of $\mathcal{C}$
		\State	$c\leftarrow$ the right-hand
		\State $r\leftarrow \max(\nu(y),\nu(z))+c$
		\If {$r < L$}
			\State $r\leftarrow -\infty$
		\EndIf
		\If {$r < \nu(x) $}
			\State $\nu(x)\leftarrow r$
			\State $\text{append}(Q,x)$
		\EndIf
	\end{algorithmic}
\end{algorithm}


\chapter{On Random Mean Payoff Graphs}
\label{appendix:RandomGraphs}


\section*{Introduction}
This appendix deals with some random properties of \acrshort{mpg}. The results offered by the appendix are used in the implementation and complexity analysis used in section \ref{section:MPG:Generation} of chapter \ref{section:Dataset}.

\section{Sinkless $\mathcal{D}(n,p)$ Graph}

\subsection{Property}

Let $\mathtt{P}$ be the property\footnote{Formally, a property is a just a set of graphs. In practice, it is a set that has desirable ``properties".} ``Graph has not sink".
\newline This property is increasing in the sense that:
$$
\forall H \ \text{spanning subgraph of}\ G, \quad H \in \mathtt{P}\implies G\in\mathtt{P}
$$
As a consequence:
$$
\forall n\in\mathbb{N},p,p'\in[0,1] / \quad p\le p',\quad \mathscr{P}(\mathcal{D}(n,p)\in \mathtt{P}) \le \mathscr{P}(\mathcal{D}(n,p')\in \mathtt{P})
$$
We will be interested in two properties:
\begin{itemize}
	\item The property ``Vertex $v$ has no sinks". We denote it by $\text{NoSink}(v)$.
	\item The property ``Graph $G$ has no sinks at all". We denote it by $\text{Sinkless}(G)$.
\end{itemize}

\subsection{Basic Comparison with Normal $\mathcal{D}(n,p)$}
We will calculate the expected value of $\deg v.$ By applying the law of total expectancy:
\begin{align*}
	\mathbb{E}[\deg v]&= \mathbb{E}[\deg v\mid \deg v > 0]\times \mathscr{P}(\deg v> 0) + \mathbb{E}[\deg v \mid \deg v = 0] \times \mathscr{P}(\deg v=0) \\
	&=  \mathbb{E}[\deg v\mid \text{Sinkless}(G)]\times \mathscr{P}(\text{NoSink}(v))
\end{align*}
With that:
\begin{align*}
	\mathbb{E}[\deg v \mid \text{Sinkless}(G)] &= \frac{\mathbb{E}[\deg v]}{\mathscr{P}(\text{NoSink}(v))}=\frac{np}{1-(1-p)^n} \le \frac{np}{1-e^{-1}} \\
	\mathbb{E}[\lvert \mathcal{E} \rvert] &=\sum_{v\in V} 	\mathbb{E}[\deg v \mid \text{Sinkless}(G)]= \frac{n^2p}{1-(1-p)^n} \le \frac{n^2p}{1-e^{-1}}
\end{align*}
This shows that the conditional distribution does inflict a small multiplicative bias on the expected number of edges and expected degree.
\newline This serves as an evidence that $\mathcal{D}^S(n,p)$ is similar enough to $\mathcal{D}(n,p)$

\subsection{Property Probability}
\label{section:RandomGraphs:PropertyProbability}
\begin{itemize}
\item Let $G\sim \mathcal{D}(n,p)$
\item Let $v$ a vertex of $G$
\end{itemize}
The probability that $\text{NoSink}(v)$ occurs is:
\begin{align*}
\mathcal{P}(\text{NoSink}(v))&=1- \mathcal{P}(\Adj v=\varnothing) \\ 
&=1-\mathcal{P}(\deg v=0)  \\
&= 1-(1-p)^n  
\end{align*}
Now, it is clear that the sequence of events $(\text{NoSink}(v))_{v\in V}$ is independent.
\newline With that, the probability that the whole graph is sinkless is:
\begin{align}
	\mathcal{P}(\text{Sinkless}(G))&=\mathcal{P}(\Adj v\ne \varnothing\quad\forall v \in V) \nonumber\\
	&=\mathcal{P}\left(\bigwedge_{v\in V} \text{NoSink}(v)\right) \nonumber\\
	&=\prod_{v\in V}\mathcal{P}(\text{NoSink}(v)) \nonumber \\
	&=\left(1-(1-p)^n\right)^n \label{eqn:SinklessProbability}
\end{align}
\subsection{Asymptotic Analysis For Dense $\mathcal{D}(n,p)$}
Let $c>0.$ We have for large enough $n$:
$$
(1-p)^n \le \frac{c}{n}
$$
Which implies:
$$
(1-\tfrac{c}{n})^n \le (1-(1-p)^n)^n \le 1
$$
If we take the limit, we have:
$$
  e^{-c}\le \lim_{n\rightarrow +\infty}  (1-(1-p)^n)^n \le 1 \quad \forall c>0 
$$
By tending $c$ to $0$, we get:
$$
\lim_{n\rightarrow +\infty} (1-(1-p)^n)^n=1
$$
\subsection{Asymptotic Analysis For Sparse $\mathcal{D}(n,p)$}
Let:

\begin{align*}
	f:\mathbb{R}_+^*\times \mathbb{R}_+\times \mathbb{R} & \rightarrow \mathbb{R}_+\\
	x,k,c&\rightarrow (1-g(x,k,c))^x\\
	g:\mathbb{R}_+^*\times \mathbb{R}_+\times \mathbb{R} & \rightarrow \mathbb{R}_+\\
	x,k,c&\rightarrow \left(1-\frac{k \ln x+c}{x}\right)^x
\end{align*}

By construction, $f(n,k,c)$ is the probability of a graph following $\mathcal{G}(n,\tfrac{k\ln n+c}{n})$ to contain no sink.

We have:

\begin{align*}
	\ln g(k,x,c)&=x\ln\left(1-\frac{k\ln x+c}{x}\right)\\
	&=-k\ln x-c -\frac{(k(\ln x)+c)^2}{2x}+o\left(\frac{(\ln x)^3}{x^2}\right)\\
\end{align*}
By applying the exponential function to both sides:
\begin{align*}
	g(x,k,c)&=\exp\left(-k\ln x-c -\frac{(k\ln x+c)^2}{2x}+o\left(\frac{(\ln x)^3}{x^2}\right)\right) \\
	&=\frac{e^{-c}}{x^k}\times e^{\frac{-(k\ln x+c)^2}{2x}+o\left(\frac{(\ln x)^3}{x^2}\right)}\\
	&=\frac{e^{-c}}{x^k}\left(1-\frac{(k \ln x+c)^2}{2x}+o\left(\frac{(\ln x)^3}{x^2}\right)\right)  \\
	&=\frac{e^{-c}}{x^k}-e^{-c}\frac{k^2(\ln x)^2}{2x^{k+1}}+o\left(\frac{(\ln x)^3}{x^{k+2}}\right)\\
	&=\frac{e^{-c}}{x^k}+o\left(\frac{1}{x^k}\right)\\
	\implies 1- g(x,k,c)&=1-\frac{e^{-c}}{x^{k}} +o\left(\frac{1}{x^k}\right)
\end{align*}
Now, we appy $\ln$ to both sides, and multiply by $x:$ 
\begin{align*}
	x\ln(1-g(x,k,c))&= -\frac{e^{-c}}{x^{k-1}}+o\left(\frac{1}{x^{k-1}}\right) \\
	&\sim -\frac{e^{-c}}{x^{k-1}}  \\
\end{align*}
Finally, we apply the exponential function to both sides, to get the desired estimation of $f:$
\begin{equation}
	\label{eqn:SinklessAsymptotic}
	 f(x,k,x) = e^{-\frac{e^{-c}}{x^{k-1}}+o(\frac{1}{x^{k-1}})}
\end{equation}

Now with that:
$$
\lim_{x\rightarrow +\infty} x\ln (1-g(x,k,c))=\begin{cases}
	-\infty  & \text{if} \ k\in[0,1[ \\
	-e^{-c} & \text{if}\ k=1  \\
	0 & \text{otherwise if}\ k\in \mathopen]1,+\infty\mathclose[ 
\end{cases}
$$
Finally, we can conclude that:
\begin{equation}
	\label{eqn:SinklessProbabilityLimit}
	\lim_{x\rightarrow +\infty} f(x,k)\begin{cases}
		0  & \text{if} \ k\in \mathopen[0,1\mathclose[ \\
		e^{-e^{-c}} & \text{if}\ k=1  \\
		1 & \text{otherwise if}\ k\in \mathopen]1,+\infty\mathclose[ 
	\end{cases}
\end{equation}

\section{Verification of distribution properties}
In section \ref{section:Dataset:ProposedDistributions:Properties} of chapter \ref{chapter:Dataset}, we required our \acrshort{mpg} distributions to be fair and symmetric.

We have proposed some useful weight distributions and graph distributions in table \ref{table:Distributions}. Then, we have also boldly claimed that any combination of both types of such distributions will give a \acrshort{mpg} distribution verifying both properties.

In this section, we will prove that claim. We will first start by the following notations:
\begin{itemize}
	\item Let $\mathcal{D}$ be a graph distribution.
	\item Let $\mathcal{W}$ be a symmetric weight distribution. 
	\item Let $D=(V,E)\sim \mathcal{D},$ and let $n=\lvert V \rvert$
	\item Let $W\sim \mathcal{W}$
	\item Let $s\sim \mathcal{U}(V)$
	\item Let $p\sim \mathcal{U}(\PlayerSet)$ where $\PlayerSet=\{\Max,\Min\}$
	\item Let $G=(V,E,W,s,p)$
\end{itemize}

\subsection{Symmetric}
This is immediate as $\mathcal{W}$ is symmetric by construction.
\subsection{Fairness}
\begin{align*}
	\mathscr{P}(\Max \ \text{wins in} \ G \ \text{with optimal play}) &=
	\mathscr{P}(v(G) > 0)  \\ 
	&= \mathscr{P}(v(\bar{G})<0) \\
	&= \mathscr{P}(v(V,E,-W,s,\bar{p})<0) \\
	&= \mathscr{P}(v(V,E,W,s,\bar{p})<0) \quad \text{as} \ W\sim \mathcal{W} \ \text{and}\ \mathcal{W} \ \text{is symmetric} \\  
	&= \mathscr{P}(v(V,E,W,s,p)<0) \quad \text{as} \ p \sim \mathcal{P} \\
	&=\mathscr{P}(v(G)<0) \\
		&=\mathscr{P}(\Min \ \text{wins in} \ G \ \text{with optimal play}) 
\end{align*} 
\subsection{Conclusion}
We conclude the result for any combination of distributions following the table \ref{table:Distributions} as they follow the hypotheses.

\section{Expected Mean Payoff}
\subsection{Definition}
\begin{itemize}
	\item Let $\mathcal{G}=(\mathcal{V},\mathcal{E})$ be a mean-payoff game
	\item For $u\in\mathcal{V},$ Let $\mathscr{P}(u)$ be the set of probability distributions over the set $\text{Adj}(u)$
	\item We define a fractional strategy as a function $\Phi\in \mathscr{P}$

\end{itemize}


\subsection{Matrix Form}
\begin{itemize}
	\item Let $n=\lvert \mathcal{V}\rvert$
	\item Let $u_1,\dots,u_n$ an enumeration of elements of $\mathcal{V}$
	A fractional strategy can be represented as a matrix $A$ such that:
	$$
	\mathcal{P}(\Phi(u_i)=u_j)=A_{i,j}
	$$
\end{itemize}


\chapter{On Probabilistic Strategies}
\label{appendix:Probabilistic:Strategies}
In the previous chapters, we gave a rough analysis of graph generation.
\newline In this chapter, we will dive into a more detailed analysis.

\section{Markovian Nature}
\subsection{Fixing $\Pi^{\Player}$}
\subsection{Fixing both $\Pi^{\Max}$ and $\Pi^{\Min}$}
\label{section:ProbabilisticStrategies:MRP}


\section{Expected Reward of a MRP}
\subsection{Markov Reward Process}
\subsubsection{Definition}
Let $\mathcal{R}=(V,E,W,A,\gamma)$ be a discrete markov reward model:
\begin{itemize}
	\item $V$ is a finite set of states
	\item $E\subseteq V\times V$ is a finite set of edges
	\item $W:E\rightarrow \mathbb{R}$ is the weights function
	\item  $A:E\rightarrow [0,1]$ is the transition function, satifying:
	\begin{align}
		\sum_{v\in\Adj u} A(u,v) &= 1 
	\end{align}
	\item $\gamma\in [0,1]$ is the discount factor
\end{itemize}
Such process models a markov chain where from a given state $u$, a transition $(u,v)\in E$ occurs with probability $A(u,v),$ and gives a reward of $W(u,v).$
\newline 
In this section, we will extend both $A$ and $W$ to $V\times V$ by requiring that:
$$
\forall e \notin E, \quad A(e)=W(e)=0
$$
\subsubsection{Execution and Award}
The execution is formalized as follow:
\begin{itemize}
	\item Fix $X_0=s\in V$
	\item For $n\in\mathbb{N}^*$, $X_n$ will be chosen randomly from the discrete set $\Adj X_{n-1}$ using probabilities from $A$
\end{itemize} 
The cumulative (discounted) reward\footnote{``reward" and ``payoff" here as synonymous. We use the word ``reward"} $R_\gamma(s)$ is defined as:
\begin{align}
	\label{eqn:TotalReward}
	R_\gamma(s)=\sum_{n\in\mathbb{N}}\gamma^n W(X_n,X_{n+1})
\end{align}
This term converges for $\gamma \in\mathopen [0,1\mathclose[.$


For undiscounted rewards\footnote{Which means $\gamma=1.$}, such reward may not converge. 
\newline For that the average time reward $\bar{R}(s)$ starting from $s$, is defined as follow:
\begin{align}
	\bar{R}(s)=\lim_{n\rightarrow +\infty} \frac{1}{n} \sum_{k=0}^{n-1} W(X_{k},X_{k+1})
\end{align}



\subsection{Expected discounted reward}
The discounted reward is used to calculate the rewards, where in each step, the weight of that rewards decay by $\gamma.$
\newline Here we will assume $\gamma \in\mathopen[0,1\mathclose[.$
\begin{align*}
	\Expected{R_\gamma(u)}&=
	\Expected{\ConditionalExpected{R_\gamma(X_0)}{X_1}}\\
	\Expected{R_\gamma(u)} & =\sum_{v\in \Adj \ u} \mathcal{P}(X_1=v\mid X_0=u)\times (W(u,v)+ \gamma \Expected{R_\gamma(v)}) \\
	&=\sum_{v\in V}  A(u,v) \times (W(u,v)+\gamma \Expected{R_\gamma(v)})
\end{align*}
Now, by considering $S_n,A,W$ as matrices, the equation can be simplified to\footnote{$\gamma$ has to be less then $1,$ as otherwise we cannot generally invert the matrix $I-\gamma A.$} :
\begin{align*}
	\Expected{R_\gamma} 
	&=\gamma A\Expected{R_\gamma}+(A\odot W)\ones
\end{align*}
Here: 
\begin{itemize}
	\item $\odot$ is the point-wise matrix product, and
	\item $\ones=(1,\dots,1)^T$ is the vector of ones.
\end{itemize}
With that, the discounted reward is calculated as follow:
\begin{align}
	\Expected{R_\gamma}&=(I-\gamma A)^{-1}  (A\odot W)\ones
\end{align}
\subsection{Expected average-time reward}
\label{section:ProbabilisticStrategies:AverageTimeReward}
While the dicounted reward converges for all $\gamma\in\mathopen[0,1\mathclose [.$ It may fail to converge in general for $\gamma=1.$ 
\newline For that, we will use the average-time reward. Such metric is more informing as it does not prioritize earlier rewards. Instead, all the rewards have the same weight and the mean calculation.
\newline In the other hand, while the intuition behind such metric is clear, it is more challenging to analyse its convergence, and calculate it directly. And this is exactly what we will do next. 
\subsubsection{Deriving Formula}
\begin{align*}
	\Expected{S_n(u)}&= \Expected{\ConditionalExpected{S_n(X_0)}{X_1}} \\
	\Expected{S_n(u)} & =\sum_{v\in \Adj \ u} \mathcal{P}(X'=v\mid X=u)\times (W(u,v)+ \Expected{S_{n-1}(v)}) \\
	&=\sum_{v\in V}  A(u,v) \times (W(u,v)+\Expected{S_{n-1}(v)})
\end{align*}
Now, by considering $S_n,A,W$ as matrices, the equation can be simplified to:
\begin{align*}
	\Expected{S_n} 
	&=A\Expected{S_{n-1}}+(A\odot W)\ones\\
	&=A\Expected{S_{n-1}}+(A\odot W)\ones\\
	&= \sum_{k=0}^{n-1}A^k(A\odot W)\ones+A^n\Expected{S_0} \\
	&=\sum_{k=0}^{n-1}A^k(A\odot W)\ones\\
\end{align*}
Now, by taking the mean, and then the limit, we have:
\begin{align}
	\label{eqn:ExpectedAverageReward}
	\Expected{\bar{R}}&=\lim_{n\rightarrow +\infty}\frac{1}{n}\sum_{k=0}^{n-1}A^k(A\odot W)\ones
\end{align}
\subsubsection{Convergence}
It is not trivial that the right-hand side of \eqref{eqn:ExpectedAverageReward} converges.
\newline \citeauthor{AverageTimeRewardStochastic}\cite{AverageTimeRewardStochastic} proved that such limits exists and calculated its value. We will reformulate his results in the following theorem.

\begin{theorem}
\label{theorem:AverageTimeRewardConvergence}
For any stochastic matrix $A,$ the limit $\displaystyle \lim_{n\rightarrow +\infty} \frac{1}{n}\sum_{k=0}^{n-1}A^k$ exists, and is equal to the \textbf{projection} matrix $T$ \textbf{uniquely} defined as follow:
\begin{itemize}
	\item $\ImageSet  T = \ker (\Id-A)$
	\item $\ImageSet T^H= \ker(\Id -A^H)$
\end{itemize}   	
\end{theorem}
Now theorem \ref{theorem:AverageTimeRewardConvergence} gives a straightforward construction of $T.$ We only have to build a projection matrix such that:
\begin{itemize}
	\item It spans $\ker(\Id -A)$
	\item It cancels at $\ker T = \ker(\Id -A^H)^\perp = \ImageSet (\Id - A)$
\end{itemize} 
Furthermore, \citeauthor{AverageTimeRewardStochastic}\cite{AverageTimeRewardStochastic} gave a closed-form expression that applies to such projection $T$:
\begin{align}
T =\lim_{n\rightarrow +\infty}\frac{1}{n}\sum_{k=0}^{n-1}A^k =X(Y^HX)^{-1}Y^H
\end{align}
Where:
\begin{itemize}
	\item $X$ is the matrix whose column vectors span $\ImageSet T$
	\item $Y$ is the matrix whose column vectors span $\ImageSet T^H$
\end{itemize}
\section{Evaluation of probabilistic strategies}
\subsection{Definition}
\begin{itemize}
	\item Let $\mathcal{G}=(\mathcal{V},\mathcal{E})$ be a mean-payoff game
	\item For $u\in\mathcal{V},$ Let $\mathscr{P}(u)$ be the set of probability distributions over the set $\text{Adj}(u)$
	\item We define a fractional strategy as a function $\Phi\in \mathscr{P}$
	
\end{itemize}


\subsection{Matrix Form}
\begin{itemize}
	\item Let $n=\lvert \mathcal{V}\rvert$
	\item Let $u_1,\dots,u_n$ an enumeration of elements of $\mathcal{V}$
	A fractional strategy can be represented as a matrix $A$ such that:
	$$
	\mathcal{P}(\Phi(u_i)=u_j)=A_{i,j}
	$$
\end{itemize}


\nocite{*}

\printbibliography[title=Bibliography]
\addcontentsline{toc}{chapter}{Bibliography}







\end{document}

% End of document
